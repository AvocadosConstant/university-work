\documentclass{article}

\usepackage{fancyhdr}
\usepackage{extramarks}
\usepackage{amsmath}
\usepackage{amsthm}
\usepackage{amsfonts}
\usepackage{tikz}
\usepackage[plain]{algorithm}
\usepackage{algpseudocode}
\usepackage{enumerate}
\usepackage{mathtools}

\usetikzlibrary{automata,positioning}

%
% Basic Document Settings
%

\topmargin=-0.45in
\evensidemargin=0in
\oddsidemargin=0in
\textwidth=6.5in
\textheight=9.0in
\headsep=0.25in

\linespread{1.1}

\pagestyle{fancy}
\lhead{\hmwkAuthorName}
\chead{\hmwkClass\ (\hmwkClassInstructor\ \hmwkClassTime): \hmwkTitle}
\rhead{\firstxmark}
\lfoot{\lastxmark}
\cfoot{\thepage}

\renewcommand\headrulewidth{0.4pt}
\renewcommand\footrulewidth{0.4pt}

\setlength\parindent{0pt}

%
% Create Problem Sections
%

\newcommand{\enterProblemHeader}[1]{
    \nobreak\extramarks{}{Problem \arabic{#1} continued on next page\ldots}\nobreak{}
    \nobreak\extramarks{Problem \arabic{#1} (continued)}{Problem \arabic{#1} continued on next page\ldots}\nobreak{}
}

\newcommand{\exitProblemHeader}[1]{
    \nobreak\extramarks{Problem \arabic{#1} (continued)}{Problem \arabic{#1} continued on next page\ldots}\nobreak{}
    \stepcounter{#1}
    \nobreak\extramarks{Problem \arabic{#1}}{}\nobreak{}
}

\setcounter{secnumdepth}{0}
\newcounter{partCounter}
\newcounter{homeworkProblemCounter}
\setcounter{homeworkProblemCounter}{1}
\nobreak\extramarks{Problem \arabic{homeworkProblemCounter}}{}\nobreak{}

%
% Homework Problem Environment
%
% This environment takes an optional argument. When given, it will adjust the
% problem counter. This is useful for when the problems given for your
% assignment aren't sequential. See the last 3 problems of this template for an
% example.
%
\newenvironment{homeworkProblem}[1][-1]{
    \ifnum#1>0
        \setcounter{homeworkProblemCounter}{#1}
    \fi
        \section{Problem \arabic{homeworkProblemCounter}}
    \setcounter{partCounter}{1}
    \enterProblemHeader{homeworkProblemCounter}
}{
    \exitProblemHeader{homeworkProblemCounter}
}

%
% Homework Details
%   - Title
%   - Due date
%   - Class
%   - Section/Time
%   - Instructor
%   - Author
%

\newcommand{\hmwkTitle}{Theory Assignment\ \#2}
\newcommand{\hmwkDueDate}{February 24, 2016}
\newcommand{\hmwkClass}{CS 375}
\newcommand{\hmwkClassInstructor}{Professor Lei Yu}
\newcommand{\hmwkClassTime}{Section B1}
\newcommand{\hmwkAuthorName}{Tim Hung}

%
% Title Page
%

\title{
    \vspace{2in}
    \textmd{\textbf{\hmwkClass:\ \hmwkTitle}}\\
    \normalsize\vspace{0.1in}\small{Due\ on\ \hmwkDueDate\ at 2:20pm}\\
    \vspace{0.1in}\large{\textit{\hmwkClassInstructor\ \hmwkClassTime}}\\
    \vspace{1in}\large{
        I have done this assignment completely on my own. I have not copied it, nor have I given my solution to anyone else. I understand that if I am involved in plagiarism or cheating I will have to sign an official form that I have cheated and that this form will be stored in my official university record. I also understand that I will receive a grade of 0 for the involved assignment for my first offense and that I will receive a grade of “F” for the course for any additional offense.
    }
    \vspace{1in}
}

\author{\textbf{\hmwkAuthorName}}
\date{}

\renewcommand{\part}[1]{\textbf{\large Part \Alph{partCounter}}\stepcounter{partCounter}\\}

%
% Various Helper Commands
%

% Useful for algorithms
\newcommand{\alg}[1]{\textsc{\bfseries \footnotesize #1}}

% For derivatives
\newcommand{\deriv}[1]{\frac{\mathrm{d}}{\mathrm{d}x} (#1)}

% For partial derivatives
\newcommand{\pderiv}[2]{\frac{\partial}{\partial #1} (#2)}

% Integral dx
\newcommand{\dx}{\mathrm{d}x}

% Alias for the Solution section header
\newcommand{\solution}{\textbf{\large Solution}}

% Probability commands: Expectation, Variance, Covariance, Bias
\newcommand{\E}{\mathrm{E}}
\newcommand{\Var}{\mathrm{Var}}
\newcommand{\Cov}{\mathrm{Cov}}
\newcommand{\Bias}{\mathrm{Bias}}

\begin{document}

\maketitle

\pagebreak

\begin{homeworkProblem} 
(24 points) Use the Master theorem to solve the following recurrences (show necessary steps to justify your answer).

    \begin{enumerate}[a)]
        \item
            $T(n)=3T(\frac{n}{4})+n$
        \item
            $T(n)=2T(\frac{n}{4})+\sqrt{n}\lg{n}$

            \textbf{Solution}
            \[
                \begin{aligned}
                    a = 2 &, && b = 4, && f(n) = \sqrt{n}\log(n)\\
                \end{aligned}\\
            \]
            \[
                f(n)  \in \Theta(n^{\log_{b}a}\log^{k}n)
            \]
            \[
                n^{\log_{b}a} = n^{\log_{4}2} = n^{\frac{1}{2}} = \sqrt{n}
            \]
            \[
                \sqrt{n}\log(n) \in \Theta(\sqrt{n}\log^{1}n)
            \]
            \[
                T(n) = \Theta(\sqrt{n}\log^{1+1}n) = \Theta(\sqrt{n}\log^{2}n)
            \]
        \item
            $T(n)=5T(\frac{n}{2})+n^2$
    \end{enumerate}

\textbf{Solution}\\

\end{homeworkProblem}

\begin{homeworkProblem}
(20 points) Solve the recurrence 
\begin{equation}
    T(n) =
        \begin{cases}
            \Theta(1) &\text{for}\ n\leq1\\
            T(\frac{n}{4}) + T(\frac{3n}{4}) + n &\text{otherwise}
        \end{cases}
    \end{equation}
using the recursion tree method. Draw the recursion tree and show the aggregate instruction counts for the following levels (0th, 1st, and last levels), and derive the $\Theta$ growth class for $T(n)$ with justifications.

\end{homeworkProblem}

\begin{homeworkProblem}
(10 points) Use the substitution method to prove that $T(n) = T(n–1) + n \in O(n^2)$. You can assume $T(1) = 1$.
\end{homeworkProblem}

\begin{homeworkProblem}
(21 points) Assume that you are given an array of $n(n\geq1)$ elements sorted in non-descending order. Design a ternary search function that searches the array for a given element x by applying the divide and conquer strategy. Hint: extend the binary search example introduced in the class - divide the array into three subarrarys where each subarray has $\frac{n}{3}$ (or almost $\frac{n}{3}$) elements). \\

Your answer should contain four parts:\\

a) Briefly describe the divide, conquer, and combine steps\\
b) Clearly define the recursive function ternarySearch(x, A, left, right), where x is the element to search for in the array A with starting index left and ending index right\\
c) Clearly define the recursive time complexity function T(n) for ternarySearch(x, A, left, right)\\
d) Solve the recursive T(n) by the master theorem\\
\end{homeworkProblem}

\pagebreak

\begin{homeworkProblem}
(25 points) The median of a list of numbers is its 50th percentile: half the numbers are bigger than it, and half are smaller. For instance, the median of [45, 1, 10, 30, 25] is 25, since this is the middle element when the numbers are arranged in order. If the list has even length, there are two choices for what the middle element could be, in which case we pick the smaller of the two. For example, the median of [45, 1, 10, 30] is 10.\\

Computing the median of n numbers is easy: just sort them. The drawback of this approach is that this takes O(n log n) time, whereas we would ideally like something linear. We have reason to be hopeful, because sorting is doing far more work than what we really need - we just want the middle element and don't care about the relative ordering of the rest of them. Can we develop a recursive solution for deciding the median of a list of numbers?\\
When looking for a recursive solution, it is paradoxically often easier to work with a more general version of the problem. In our case, the generalization we will consider is selection.\\

SELECTION\\
Input: A list of numbers S; an integer k\\
Output: The kth smallest element of S\\
For instance, if k = 1, the minimum of S is sought, whereas if k = ceiling(|S|/2), it is the median.\\

Develop a divide-and-conquer approach to selection (and hence a solution for the finding median problem). Hint: for any number v, imagine splitting list S into three categories: elements smaller than v, those equal to v (there might be duplicates), and those greater than v. \\

In your answer, show the following:\\

a) Briefly describe the divide, conquer, and combine steps\\
b) Clearly define the recursive function for selection(S, k); (note: this is not the function for the time complexity of the selection function.)\\
c) Analyze the best case and worst case time complexity of this approach given input size n.\\
\end{homeworkProblem}

\pagebreak

\begin{homeworkProblem}
\textbf{Bonus Question} (20 points):\\
We know that the master theorem does not apply to the recursive function T(n) = 2T(n/2) + n / lg n.
Use the recursion tree method to solve this recursion. Draw the recursion tree and show the aggregate instruction counts for the following levels (0th, 1st, and last levels), and derive the Θ growth class for T(n) with justifications.
\end{homeworkProblem}

\end{document}
