\documentclass{article}

\usepackage{fancyhdr}
\usepackage{extramarks}
\usepackage{amsmath}
\usepackage{amsthm}
\usepackage{amsfonts}
\usepackage{tikz}
\usepackage[plain]{algorithm}
\usepackage{algpseudocode}
\usepackage{enumerate}
\usepackage{mathtools}
\usepackage{forest}
\usepackage{adjustbox}
\usepackage[table]{colortbl}
\usepackage{wrapfig}


\usetikzlibrary{automata,positioning,backgrounds}

%
% Basic Document Settings
%

\topmargin=-0.45in
\evensidemargin=0in
\oddsidemargin=0in
\textwidth=6.5in
\textheight=9.0in
\headsep=0.25in

\linespread{1.1}

\pagestyle{fancy}
\lhead{\hmwkAuthorName}
\chead{\hmwkClass\ (\hmwkClassInstructor\ \hmwkClassTime): \hmwkTitle}
\rhead{\firstxmark}
\lfoot{\lastxmark}
\cfoot{\thepage}

\renewcommand\headrulewidth{0.4pt}
\renewcommand\footrulewidth{0.4pt}

\setlength\parindent{0pt}

%
% Create Problem Sections
%

\newcommand{\enterProblemHeader}[1]{
    \nobreak\extramarks{}{Problem \arabic{#1} continued on next page\ldots}\nobreak{}
    \nobreak\extramarks{Problem \arabic{#1} (continued)}{Problem \arabic{#1} continued on next page\ldots}\nobreak{}
}

\newcommand{\exitProblemHeader}[1]{
    \nobreak\extramarks{Problem \arabic{#1} (continued)}{Problem \arabic{#1} continued on next page\ldots}\nobreak{}
    \stepcounter{#1}
    \nobreak\extramarks{Problem \arabic{#1}}{}\nobreak{}
}

\setcounter{secnumdepth}{0}
\newcounter{partCounter}
\newcounter{homeworkProblemCounter}
\setcounter{homeworkProblemCounter}{1}
\nobreak\extramarks{Problem \arabic{homeworkProblemCounter}}{}\nobreak{}

%
% Homework Problem Environment
%
% This environment takes an optional argument. When given, it will adjust the
% problem counter. This is useful for when the problems given for your
% assignment aren't sequential. See the last 3 problems of this template for an
% example.
%
\newenvironment{homeworkProblem}[1][-1]{
    \ifnum#1>0
        \setcounter{homeworkProblemCounter}{#1}
    \fi
        \section{Problem \arabic{homeworkProblemCounter}}
    \setcounter{partCounter}{1}
    \enterProblemHeader{homeworkProblemCounter}
}{
    \exitProblemHeader{homeworkProblemCounter}
}

%
% Homework Details
%   - Title
%   - Due date
%   - Class
%   - Section/Time
%   - Instructor
%   - Author
%

\newcommand{\hmwkTitle}{Theory Assignment\ \#4}
\newcommand{\hmwkDueDate}{May 2, 2016}
\newcommand{\hmwkClass}{CS 375}
\newcommand{\hmwkClassInstructor}{Professor Lei Yu}
\newcommand{\hmwkClassTime}{Section B1}
\newcommand{\hmwkAuthorName}{Tim Hung}

%
% Title Page
%

\title{
    \vspace{2in}
    \textmd{\textbf{\hmwkClass:\ \hmwkTitle}}\\
    \normalsize\vspace{0.1in}\small{Due\ on\ \hmwkDueDate\ at 2:20pm}\\
    \vspace{0.1in}\large{\textit{\hmwkClassInstructor\ \hmwkClassTime}}\\
    \vspace{1in}\large{
        I have done this assignment completely on my own. I have not copied it, nor have I given my solution to anyone else. I understand that if I am involved in plagiarism or cheating I will have to sign an official form that I have cheated and that this form will be stored in my official university record. I also understand that I will receive a grade of 0 for the involved assignment for my first offense and that I will receive a grade of “F” for the course for any additional offense.
    }
    \vspace{1in}
}

\author{\textbf{\hmwkAuthorName}}
\date{}

\renewcommand{\part}[1]{\textbf{\large Part \Alph{partCounter}}\stepcounter{partCounter}\\}

%
% Various Helper Commands
%

% Useful for algorithms
\newcommand{\alg}[1]{\textsc{\bfseries \footnotesize #1}}

% For derivatives
\newcommand{\deriv}[1]{\frac{\mathrm{d}}{\mathrm{d}x} (#1)}

% For partial derivatives
\newcommand{\pderiv}[2]{\frac{\partial}{\partial #1} (#2)}

% Integral dx
\newcommand{\dx}{\mathrm{d}x}

% Alias for the Solution section header
\newcommand{\solution}{\textbf{\large Solution}}

% Probability commands: Expectation, Variance, Covariance, Bias
\newcommand{\E}{\mathrm{E}}
\newcommand{\Var}{\mathrm{Var}}
\newcommand{\Cov}{\mathrm{Cov}}
\newcommand{\Bias}{\mathrm{Bias}}

\begin{document}

\maketitle

\pagebreak

\begin{homeworkProblem} 
(20\%) A set \{3, 4, 5, 8\} is given. For the set, find every subset that sums to S = 13. Find the subsets via the backtracking algorithm. In your solution, draw a pruned state space tree. For each node in the tree, show its current subset sum and its upper bound of the sum (i.e., weightSoFar + totalPossibleLeft). Number the nodes in the sequence of visiting them. Also, identify the node that represents the solution found at the end of the search.
\end{homeworkProblem} 

\begin{homeworkProblem}
    \begin{wrapfigure}{r}{0pt}
        \begin{tabular}{c c c c}
            i   &   $p_i$ & $w_i$ & $\frac{p_i}{w_i}$ \\ \hline
            1   &   \$10    &   5   &    \$2    \\
            2   &   \$30    &   2   &    \$15   \\
            3   &   \$40    &   5   &    \$8    \\
            4   &   \$30    &  10   &    \$3    \\
        \end{tabular}
    \end{wrapfigure}
    (15\%) When the capacity of the knapsack is 15, solve the following 0-1 knapsack problem using the backtracking algorithm that uses the optimal fractional knapsack algorithm to compute the upper bound of the profit.\\



\end{homeworkProblem}

\begin{homeworkProblem}
(15\%) For the same problem in Question 2, solve it using the best-first-search branch and bound algorithm. Follow the same instructions above to produce your solution. 
\end{homeworkProblem}

\begin{homeworkProblem}
    \begin{wrapfigure}{r}{0pt}
        \begin{tikzpicture}[auto, node distance=2cm, every loop/.style={},
            thick,main node/.style={circle,draw,font=\sffamily\Large\bfseries}]

            \node[main node] (a) {a};
            \node[main node] (b) [below left of=a] {b};
            \node[main node] (c) [below right of=a] {c};
            \node[main node] (d) [below right of=b] {d};

            \path[every node/.style={font=\sffamily\small}]
            (a) edge node [above right] {1} (c)
            (b) edge node [above left] {8} (a)
                edge node {3} (c)
                edge node [below left] {2} (d)
            (c) edge node [below right] {5} (d);
        \end{tikzpicture}
    \end{wrapfigure}

    (15\%) Apply Prim's algorithm for finding a minimum spanning tree for the following graph. Start with node a.  Show the steps by filling out the following table (see the example on slide 31 of lecture 25). Show the selected tree nodes in the first column of the table, for each of the rest of the nodes, show its minimum distance D to the current tree and its nearest node in the current tree, in the remaining columns. \\

    \begin{tabular}{|c|c|c|c|c|}\hline
        Node added  & D(a)  & D(c)  &D(d)\\ \hline \hline
    \end{tabular}


\end{homeworkProblem}

\begin{homeworkProblem}
    \begin{wrapfigure}{r}{0pt}
        \begin{tikzpicture}[auto, node distance=1.5cm,
            thick,main node/.style={circle,draw,font=\sffamily\Large\bfseries}]

            \node[main node] (a) {a};
            \node[main node] (b) [below left of=a, xshift=-.5cm] {b};
            \node[main node] (c) [below of=b] {c};
            \node[main node] (d) [below right of=c, xshift=.5cm] {d};
            \node[main node] (e) [below right of=a, xshift=.5cm] {e};
            \node[main node] (f) [below of=e] {f};
            \node[main node] (g) [right of=e] {g};
            \node[main node] (h) [right of=f] {h};

            \path[every node/.style={font=\sffamily\small}]
            (a) edge node [above left] {6} (b)
                edge node {4} (e)
            (b) edge node {5} (e)
                edge node [left] {14} (c)
                edge [bend left] node {10} (d)
            (c) edge node [below left] {3} (d)
            (d) edge node [below right] {8} (f)
            (e) edge node {9} (g)
                edge node {2} (f)
            (f) edge node {15} (h)
                ;
        \end{tikzpicture}
    \end{wrapfigure}
    
    (10\%) Apply Kruskal’s algorithm for finding a minimum spanning tree for the following graph. In your solution, show edges picked in order and the total weight of the final minimum spanning tree.\\

    \textbf{Solution}

\end{homeworkProblem}

\pagebreak

\begin{homeworkProblem}

    \begin{wrapfigure}{r}{0pt}
        \begin{tikzpicture}[->,auto, node distance=2cm,
            thick,main node/.style={circle,draw}]

            \node[main node] (0) {s};
            \node[main node] (1) [below right of=0, xshift=1cm] {$s_1$};
            \node[main node] (2) [above right of=1, xshift=1cm] {$s_2$};
            \node[main node] (3) [above right of=0, xshift=1cm] {$s_3$};
            \node[main node] (4) [above of=3, yshift=-1cm] {$s_4$};

            \path[every node/.style={font=\sffamily\small}]
            (0) edge node {5} (1)
                edge node [below right] {3} (3)
                edge node {15} (4)
            (1) edge node {1} (2)
                edge [bend right] node {4} (3)
            (2) edge node {2} (3)
            (3) edge [bend right] node {3} (1)
            (4) edge node {8} (2)
                ;
        \end{tikzpicture}
    \end{wrapfigure}

    (15\%) Using Dijkstra’s algorithm, find the shortest path to visit each vertex starting from vertex s in the following graph. In your solution, order the vertexes in terms of their shortest path distances to the vertex s, and show the shortest path and its distance for each vertex.\\

    \textbf{Solution}\\

this here be the solution yarr

\end{homeworkProblem}


\begin{homeworkProblem}
    \begin{wrapfigure}{r}{0pt}
        \begin{tikzpicture}[->, auto, node distance=2cm, 
            thick,main node/.style={circle,draw}]

            \node[main node] (1) {1};
            \node[main node] (2) [below right of=1] {2};
            \node[main node] (4) [below left of=1] {4};
            \node[main node] (3) [below right of=4] {3};

            \path[every node/.style={font=\sffamily\small}]
            (1) edge node [above right] {8} (2)
                edge node [above left] {1} (4)
            (4) edge node [above right, xshift=.3cm] {2} (2)
                edge node [below left] {9} (3)
            (2) edge node [below right] {1} (3)
            (3) edge node [above left, yshift=.5cm] {4} (1)
            ;
        \end{tikzpicture}
    \end{wrapfigure}

    (10\%) Apply Floyd-Warshall algorithm to the following directed graph with the initial distance matrix representing the direct distance between every pair of vertices, and produce the updated distance matrices for every iteration of the algorithm. 

    \[D=\begin{bmatrix}
        0 & 8 & ? & 1\\
        ? & 0 & 1 & ?\\
        4 & ? & 0 & ?\\
        ? & 2 & 9 & 0\\
    \end{bmatrix}\]\\

    \textbf{Solution}

    \[D=\begin{bmatrix}
        0 & 8 & ? & 1\\
        ? & 0 & 1 & ?\\
        4 & ? & 0 & ?\\
        ? & 2 & 9 & 0\\
    \end{bmatrix}\]\\

\end{homeworkProblem}

\end{document}
