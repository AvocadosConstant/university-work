\sektion{1}{Decidability}

Hello welcome to the section.

\begin{corollary}
    A language is decidable if $\exists$ a non deterministic Turing Machine that recognizes it.
\end{corollary}

\begin{theorem}
    A language is Turing Recognizable if and only if some enumerator enumerates it.
\end{theorem}

\begin{theorem}
    The class of Fontext Free Languages is a proper subset of the Turing Recognizable languages.
\end{theorem}

Hilbert's 10th Problem: Given a polynomial with integer coeficients, does there exist an integer root to that polynomial.

\[D = \{p|\text{p is a polynomial over one variable}\}  \]
\[F = \{p|\text{p is a polynomial over one or more variables}\}  \]

\begin{theorem}
The class of Turing Recognizable Languages is closed under $\cup$.
\end{theorem}

\begin{proof}
Let A, B be Turing Recognizable Languages.

$\exists$ Turing Machines $M_A, M_B, L(M_A)=A, L(M_B)=B$.

We want Turing Machine M such that $L(M)=A\cup B$

On input w, M does:

1. run $M_A$ and $M_B$ in parallel on w

    -if $M_A$ or $M_B$ then halt and accept
    
    -if $M_A$ and $M_B$ then halt and reject

\begin{claim}
$L(M)=A\cup B$
\end{claim}
Let $w\in L(M)$
$w\in A\cup B$ etc.....
\end{proof}
