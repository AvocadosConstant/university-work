\sektion{1}{Coloring}

\subsektion{Vertex Coloring}

\begin{theorem}[Brook's Theorem]
    In a connected graph in which every vertex has at most $\Delta$ neighbors, the vertices can be colored with only $\Delta$ colors, except for two cases, complete graphs and cycle graphs of odd length, which require $\Delta+1$ colors.
\end{theorem}

\subsubsection{Chromatic Polynomial}
\[
    P_G (k) = P_{G_1} (k) + P_{G_2} (k) 
\]
The first coefficient is always 1.

The degree of the first term is the $(|V|)$.

The second coefficient is always $-(|E|)$.

The final (constant) coefficient is always 0.

\begin{definition}
    The chromatic polynomial of a complete graph $K_n$ on n vertices is 
    \[P_{K_n}=k(k-1)(k-2)...(k-(n-1))\]
\end{definition}

\begin{definition}
    The chromatic polynomial of a tree $T_n$ on n vertices is \[P_{T_n}=k(k-1)^{n-1}\]
\end{definition}



\subsektion{Edge Coloring}
\begin{definition}
    The chromatic index of a graph, $\chi'$, is ...
\end{definition}

    number of edges in $L(G) = \sum_{i=1}^n{\binom{d_i}{2}}$

\begin{theorem}
    $\chi'(G) = \chi(L(G))$
\end{theorem}

\begin{theorem}[Vizing's Theorem]
     The chromatic index of simple undirected graph is either $\Delta$ or $\Delta + 1$.
\end{theorem}

\begin{theorem}[K{\"o}nig's Line Coloring Theorem]
\end{theorem}

Continued line graph derivations of connected graphs

    1.  Paths

    2.  Cycles

    3.  $K_{1,3}$

    4.  All others grow

\begin{theorem}[Whitney's Theorem]
    Two connected graphs on at least 4 vertices are isomorphic if and only if their line graphs are isomorphic.
\end{theorem}




%   April 25, 2016
\subsektion{Planar Duality}

\begin{definition}[Dual Graph]
    The dual graph of a plane graph G is a graph that has a vertex for each face of G. The dual graph has an edge whenever two faces of G are separated from each other by an edge.
\end{definition}

\begin{theorem}
A graph has the same number of edges as its dual.
\end{theorem}

\begin{theorem}
A graph with n vertices and f faces has a dual with f vertices and n faces.
\end{theorem}

\begin{proposition}
    Wheels are self-dual.
\end{proposition}

\begin{proposition}
    Duality is an involution.
\end{proposition}
