\sektion{1}{Coloring}

\subsektion{Vertex Coloring}

\begin{theorem}[Brook's Theorem]
    In a connected graph in which every vertex has at most $\Delta$ neighbors, the vertices can be colored with only $\Delta$ colors, except for two cases, complete graphs and cycle graphs of odd length, which require $\Delta+1$ colors.
\end{theorem}

\subsubsection{Chromatic Polynomial}
\[
    P_G (k) = P_{G_1} (k) + P_{G_2} (k) 
\]
The first coefficient is always 1.

The degree of the first term is the $(|V|)$.

The second coefficient is always $-(|E|)$.

The final (constant) coefficient is always 0.

\begin{definition}
    The chromatic polynomial of a complete graph $K_n$ on n vertices is 
    \[P_{K_n}=k(k-1)(k-2)...(k-(n-1))\]
\end{definition}

\begin{definition}
    The chromatic polynomial of a tree $T_n$ on n vertices is \[P_{T_n}=k(k-1)^{n-1}\]
\end{definition}



\subsektion{Edge Coloring}
\begin{definition}
    The chromatic index of a graph, $\chi'$, is ...
\end{definition}
