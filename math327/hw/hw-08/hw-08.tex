\documentclass{article}

\usepackage{fancyhdr}
\usepackage{extramarks}
\usepackage{amsmath}
\usepackage{amsthm}
\usepackage{amsfonts}
\usepackage{tikz}
\usepackage{enumerate}
\usepackage{mathtools}
\usepackage{enumitem}
\usepackage{pgfplots}
\usepackage{diagbox}

%
% Basic Document Settings
%

\topmargin=-0.45in
\evensidemargin=0in
\oddsidemargin=0in
\textwidth=6.5in
\textheight=9.0in
\headsep=0.25in

\linespread{1.1}

\pagestyle{fancy}
\lhead{\hmwkAuthorName}
\chead{\hmwkClass\ (\hmwkClassInstructor\ \hmwkClassTime): \hmwkTitle}
\rhead{\firstxmark}
\lfoot{\lastxmark}
\cfoot{\thepage}

\renewcommand\headrulewidth{0.4pt}
\renewcommand\footrulewidth{0.4pt}

\setlength\parindent{0pt}

%
% Create Problem Sections
%

\newcommand{\enterProblemHeader}[1]{
    \nobreak\extramarks{}{Problem \arabic{#1} continued on next page\ldots}\nobreak{}
    \nobreak\extramarks{Problem \arabic{#1} (continued)}{Problem \arabic{#1} continued on next page\ldots}\nobreak{}
}

\newcommand{\exitProblemHeader}[1]{
    \nobreak\extramarks{Problem \arabic{#1} (continued)}{Problem \arabic{#1} continued on next page\ldots}\nobreak{}
    \stepcounter{#1}
    \nobreak\extramarks{Problem \arabic{#1}}{}\nobreak{}
}

\setcounter{secnumdepth}{0}
\newcounter{partCounter}
\newcounter{homeworkProblemCounter}
\setcounter{homeworkProblemCounter}{1}
\nobreak\extramarks{Problem \arabic{homeworkProblemCounter}}{}\nobreak{}

%
% Homework Problem Environment
%
% This environment takes an optional argument. When given, it will adjust the
% problem counter. This is useful for when the problems given for your
% assignment aren't sequential. See the last 3 problems of this template for an
% example.
%
\newenvironment{homeworkProblem}[1][-1]{
    \ifnum#1>0
    \setcounter{homeworkProblemCounter}{#1}
    \fi
    \section{Problem \arabic{homeworkProblemCounter}}
    \setcounter{partCounter}{1}
    \enterProblemHeader{homeworkProblemCounter}
    }{
    \exitProblemHeader{homeworkProblemCounter}
}

%
% Homework Details
%   - Title
%   - Due date
%   - Class
%   - Section/Time
%   - Instructor
%   - Author
%

% TODO: Replace with correct number and date
\newcommand{\hmwkTitle}{Problem Set\ \#8}
\newcommand{\hmwkDueDate}{April 05, 2017}
\newcommand{\hmwkClass}{MATH 327}
\newcommand{\hmwkClassInstructor}{Professor Mei-Hsiu Chen}
\newcommand{\hmwkAuthorName}{Tim Hung}

%
% Title Page
%

\title{
    \vspace{2in}
    \textmd{\textbf{\hmwkClass:\ \hmwkTitle}}\\
    \normalsize\vspace{0.1in}\small{Due\ on\ \hmwkDueDate\ at 2:10pm}\\
    \vspace{0.1in}\large{\textit{\hmwkClassInstructor\ \hmwkClassTime}}\\
}

\author{\textbf{\hmwkAuthorName}}
\date{}

\renewcommand{\part}[1]{\textbf{\large Part \Alph{partCounter}}\stepcounter{partCounter}\\}

%
% Various Helper Commands
%

% Useful for algorithms
\newcommand{\alg}[1]{\textsc{\bfseries \footnotesize #1}}

% For derivatives
\newcommand{\deriv}[1]{\frac{\mathrm{d}}{\mathrm{d}x} (#1)}

% For partial derivatives
\newcommand{\pderiv}[2]{\frac{\partial}{\partial #1} (#2)}

% Integral dx
\newcommand{\dx}{\mathrm{d}x}

% Alias for the Solution section header
\newcommand{\solution}{\textbf{\large Solution}}

% Probability commands: Expectation, Variance, Covariance, Bias
\newcommand{\E}{\mathrm{E}}
\newcommand{\Var}{\mathrm{Var}}
\newcommand{\Cov}{\mathrm{Cov}}
\newcommand{\Bias}{\mathrm{Bias}}

\begin{document}

\maketitle

\pagebreak

\begin{homeworkProblem}[1]

    Let $X_1, ... , X_n$ be a sample from the distribution whose density function is
    \[
        f(x) = \begin{cases}
            e^{−(x−θ )} & x \geq \theta \\
            0           & otherwise
        \end{cases}
    \]
    Determine the maximum likelihood estimator of $\theta$.\\

    \textbf{Solution} 

\end{homeworkProblem}

\begin{homeworkProblem}[3]

    Let $X_1, ... , X_n$ be a sample from a normal $\mu, \sigma^2$ population. Determine the maximum likelihood estimator of $σ^2$ when $\mu$ is known. What is the expected value of this estimator?\\

    \textbf{Solution} 
    
\end{homeworkProblem}

\begin{homeworkProblem}[5]

    Suppose that 
    $X_1, ... , X_n$ are normal with mean $\mu_1$; 
    $Y_1, ... , Y_n$ are normal with mean $\mu_2$; and 
    $W_1, ... , W_n$ are normal with mean $\mu_1 + \mu_2$. 
    Assuming that all $3n$ random variables are independent with a common variance, find the maximum likelihood estimators of $\mu_1$ and $\mu_2$.\\

    \textbf{Solution} 
    
\end{homeworkProblem}

\begin{homeworkProblem}[8]

    An electric scale gives a reading equal to the true weight plus a random error that
    is normally distributed with mean $\theta$ and standard deviation $\sigma = .1 mg$. 
    Suppose that the results of five successive weighings of the same object are as 
    follows: 3.142, 3.163, 3.155, 3.150, 3.141.

    \begin{enumerate}[label=(\alph*)]
        \item Determine a 95 percent confidence interval estimate of the true weight.
        \item Determine a 99 percent confidence interval estimate of the true weight
    \end{enumerate}
    
\end{homeworkProblem}

\begin{homeworkProblem}[9]

    The PCB concentration of a fish caught in Lake Michigan was measured by a
    technique that is known to result in an error of measurement that is normally
    distributed with a standard deviation of $.08 ppm$ (parts per million). Suppose the
    results of 10 independent measurements of this fish are

    \[ 11.2, 12.4, 10.8, 11.6, 12.5, 10.1, 11.0, 12.2, 12.4, 10.6 \]

    \begin{enumerate}[label=(\alph*)]
        \item Give a 95 percent confidence interval for the PCB level of this fish.
        \item Give a 95 percent lower confidence interval.
        \item Give a 95 percent upper confidence interval.
    \end{enumerate}
    
\end{homeworkProblem}

\begin{homeworkProblem}[11]

    Let $X_1, ... , X_n, X_{n+1}$ be a sample from a normal population having an unknown mean $\mu$ and variance $1$. Let $\overline{X_n} = \sum_{i=1}^n \frac{X_i}{n}$ be the average of the first n of them.  

    \begin{enumerate}[label=(\alph*)]
        \item What is the distribution of $X_{n+1} − \overline{X_n}$?
        \item If $\overline{X_n} = 4$, give an interval that, with 90 percent confidence, will contain the value of $X_{n+1}$.
    \end{enumerate}
    
\end{homeworkProblem}

\begin{homeworkProblem}[13]

    A sample of 20 cigarettes is tested to determine nicotine content and the average
    value observed was 1.2 mg. Compute a 99 percent two-sided confidence interval
    for the mean nicotine content of a cigarette if it is known that the standard
    deviation of a cigarette’s nicotine content is $\sigma = .2 mg$.\\

    \textbf{Solution} 

\end{homeworkProblem}

\begin{homeworkProblem}[14]

    In Problem 13, suppose that the population variance is not known in advance
    of the experiment. If the sample variance is .04, compute a 99 percent two-sided
    confidence interval for the mean nicotine content.\\

    \textbf{Solution} 

\end{homeworkProblem}

\begin{homeworkProblem}[15]

    In Problem 14, compute a value c for which we can assert “with 99 percent con-
    fidence” that c is larger than the mean nicotine content of a cigarette.\\

    \textbf{Solution} 
    
\end{homeworkProblem}

\begin{homeworkProblem}[18]

    The following are scores on IQ tests of a random sample of 18 students at a large
    eastern university.

    \[130, 122, 119, 142, 136, 127, 120, 152, 141, 132, 127, 118, 150, 141, 133, 137, 129, 142\]

    \begin{enumerate}[label=(\alph*)]
        \item Construct a 95 percent confidence interval estimate of the average IQ score of all students at the university.
    \end{enumerate}
    
\end{homeworkProblem}

\begin{homeworkProblem}[21]

    A standardized test is given annually to all sixth-grade students in the state of
    Washington. To determine the average score of students in her district, a school
    supervisor selects a random sample of 100 students. If the sample mean of these
    students’ scores is 320 and the sample standard deviation is 16, give a 95 percent
    confidence interval estimate of the average score of students in that supervisor’s
    district.\\

    \textbf{Solution} 
    
\end{homeworkProblem}

\begin{homeworkProblem}[23]

    A random sample of 300 CitiBank VISA cardholder accounts indicated a sample
    mean debt of \$1,220 with a sample standard deviation of \$840. Construct
    a 95 percent confidence interval estimate of the average debt of all cardholders.\\

    \textbf{Solution} 
    
\end{homeworkProblem}

\end{document}
