\documentclass{article}

\usepackage{fancyhdr}
\usepackage{extramarks}
\usepackage{amsmath}
\usepackage{amsthm}
\usepackage{amsfonts}
\usepackage{tikz}
\usepackage{enumerate}
\usepackage{mathtools}
\usepackage{enumitem}
\usepackage{pgfplots}
\usepackage{diagbox}

%
% Basic Document Settings
%

\topmargin=-0.45in
\evensidemargin=0in
\oddsidemargin=0in
\textwidth=6.5in
\textheight=9.0in
\headsep=0.25in

\linespread{1.1}

\pagestyle{fancy}
\lhead{\hmwkAuthorName}
\chead{\hmwkClass\ (\hmwkClassInstructor\ \hmwkClassTime): \hmwkTitle}
\rhead{\firstxmark}
\lfoot{\lastxmark}
\cfoot{\thepage}

\renewcommand\headrulewidth{0.4pt}
\renewcommand\footrulewidth{0.4pt}

\setlength\parindent{0pt}

%
% Create Problem Sections
%

\newcommand{\enterProblemHeader}[1]{
    \nobreak\extramarks{}{Problem \arabic{#1} continued on next page\ldots}\nobreak{}
    \nobreak\extramarks{Problem \arabic{#1} (continued)}{Problem \arabic{#1} continued on next page\ldots}\nobreak{}
}

\newcommand{\exitProblemHeader}[1]{
    \nobreak\extramarks{Problem \arabic{#1} (continued)}{Problem \arabic{#1} continued on next page\ldots}\nobreak{}
    \stepcounter{#1}
    \nobreak\extramarks{Problem \arabic{#1}}{}\nobreak{}
}

\setcounter{secnumdepth}{0}
\newcounter{partCounter}
\newcounter{homeworkProblemCounter}
\setcounter{homeworkProblemCounter}{1}
\nobreak\extramarks{Problem \arabic{homeworkProblemCounter}}{}\nobreak{}

%
% Homework Problem Environment
%
% This environment takes an optional argument. When given, it will adjust the
% problem counter. This is useful for when the problems given for your
% assignment aren't sequential. See the last 3 problems of this template for an
% example.
%
\newenvironment{homeworkProblem}[1][-1]{
    \ifnum#1>0
    \setcounter{homeworkProblemCounter}{#1}
    \fi
    \section{Problem \arabic{homeworkProblemCounter}}
    \setcounter{partCounter}{1}
    \enterProblemHeader{homeworkProblemCounter}
    }{
    \exitProblemHeader{homeworkProblemCounter}
}

%
% Homework Details
%   - Title
%   - Due date
%   - Class
%   - Section/Time
%   - Instructor
%   - Author
%

\newcommand{\hmwkTitle}{Problem Set\ \#4}
\newcommand{\hmwkDueDate}{February 20, 2017}
\newcommand{\hmwkClass}{MATH 327}
\newcommand{\hmwkClassInstructor}{Professor Mei-Hsiu Chen}
\newcommand{\hmwkAuthorName}{Tim Hung}

%
% Title Page
%

\title{
    \vspace{2in}
    \textmd{\textbf{\hmwkClass:\ \hmwkTitle}}\\
    \normalsize\vspace{0.1in}\small{Due\ on\ \hmwkDueDate\ at 2:10pm}\\
    \vspace{0.1in}\large{\textit{\hmwkClassInstructor\ \hmwkClassTime}}\\
}

\author{\textbf{\hmwkAuthorName}}
\date{}

\renewcommand{\part}[1]{\textbf{\large Part \Alph{partCounter}}\stepcounter{partCounter}\\}

%
% Various Helper Commands
%

% Useful for algorithms
\newcommand{\alg}[1]{\textsc{\bfseries \footnotesize #1}}

% For derivatives
\newcommand{\deriv}[1]{\frac{\mathrm{d}}{\mathrm{d}x} (#1)}

% For partial derivatives
\newcommand{\pderiv}[2]{\frac{\partial}{\partial #1} (#2)}

% Integral dx
\newcommand{\dx}{\mathrm{d}x}

% Alias for the Solution section header
\newcommand{\solution}{\textbf{\large Solution}}

% Probability commands: Expectation, Variance, Covariance, Bias
\newcommand{\E}{\mathrm{E}}
\newcommand{\Var}{\mathrm{Var}}
\newcommand{\Cov}{\mathrm{Cov}}
\newcommand{\Bias}{\mathrm{Bias}}

\begin{document}

\maketitle

\pagebreak

\begin{homeworkProblem}[21]
    Five men and five women are ranked according to their scores on an examination.  Assume that no two scores are alike and all 10! possible rankings are equally likely. Let X denote the highest ranking achieved by a woman (for instance, $X = 2$ if the top-ranked person was male and the next-ranked person was female).
    \begin{align*}
        P\{ 1\} &= .5\\
        P\{ 2\} &= .28\\
        P\{ 3\} &= .14\\
        P\{ 4\} &= .06\\
        P\{ 5\} &= .02\\
        P\{ 6\} &= .00\\
        P\{ t \geq7\} &= 0\\
    \end{align*}
    Compute the expected value of the random variable X.\\

    \textbf{Solution}\\
    \begin{align*}
        E[X] &= \sum_{t} t P\{t\}\\
             &= 1(.5) + 2(.28) + 3(.14) + 4(.06) + 5(.02) + 6(.00) + 0\\
             &= 1.83\dots
    \end{align*}
    
\end{homeworkProblem}

\begin{homeworkProblem}
    Let X represent the difference between the number of heads and the number of tails obtained when a coin is tossed n times.
    
    If the coin is assumed fair, for n = 3, what are the probabilities associated with the values that X can take on?
    \[X = -3 + 2(k), \forall k \in \mathbb{N} \leq 3\]

    \[k = 0, X = -3 + 2(0) = -3\]
    \[k = 1, X = -3 + 2(1) = -1\]
    \[k = 2, X = -3 + 2(2) = 1\]
    \[k = 3, X = -3 + 2(3) = 3\]

    Compute the exected value of the random variable X.\\

    \textbf{Solution}\\
    \[
        E[X] &= 0 \text{, because the graph of X is symmetric and centered at 0.}
    \]

\end{homeworkProblem}

\pagebreak

\begin{homeworkProblem}
    Each night different meteorologists give us the “probability” that it will rain the next day. To judge how well these people predict, we will score each of them as follows: If a meteorologist says that it will rain with probability p, then he or she will receive a score of

    \begin{center} score = \begin{cases}
        1 - (1 - p)^2   & \text{if it does rain}\\
        1 - p^2         & \text{if it does not rain}\\
    \end{cases} \end{center}

    We will then keep track of scores over a certain time span and conclude that the meteorologist with the highest average score is the best predictor of weather.  Suppose now that a given meteorologist is aware of this and so wants to maximize his or her expected score. If this individual truly believes that it will rain tomorrow with probability $p^\ast$, what value of $p$ should he or she assert so as to maximize the expected score?\\

    \textbf{Solution}

    The score function favors correct guesses. If the meteorologist is certain that it will rain with probability $p^\ast$, then to maximize his/her score, they should assert that it will rain with probability $p^\ast$.

\end{homeworkProblem}

\begin{homeworkProblem}[25]
    A total of 4 buses carrying 148 students from the same school arrive at a football stadium. The buses carry, respectively, 40, 33, 25, and 50 students. One of the students is randomly selected. Let X denote the number of students that were on the bus carrying this randomly selected student. One of the 4 bus drivers is also randomly selected. Let Y denote the number of students on her bus.

    \begin{enumerate}[label=(\alph*)]
        \item Which of $E[X]$ or $E[Y]$ do you think is larger? Why?

            E[X] should tend to be bigger. A bus driver will be on a bus regardless of how many students are on it, but the more students are on a bus, the more likely a chosen student will be on it.
        \item Compute $E[X]$ and $E[Y]$.
            \begin{align*}
                E[X] &= \sum_{t} t P\{t\}\\
                     &= 40(\frac{40}{148}) + 33(\frac{33}{148}) + 25(\frac{25}{148}) + 50(\frac{50}{148})\\
                     &= 39.3\dots
            \end{align*}

            \begin{align*}
                E[Y] &= \sum_{t} t P\{t\}\\
                     &= \frac{1}{4} 40 + \frac{1}{4} 33 + \frac{1}{4} 25 + \frac{1}{4} 50\\
                     &= 37
            \end{align*}
    \end{enumerate}
\end{homeworkProblem}

\pagebreak

\begin{homeworkProblem}[27]
    The density function of X is given by

    \begin{center} f(x) = \begin{cases}
        a + bx^2 & 0 \leq x \leq 1\\
        0        & \text{otherwise}\\
    \end{cases} \end{center}

    If $E[X]$ = $\frac{3}{5}$, find $a$ and $b$.

    \textbf{Solution}

    \begin{align*}
        E[X] = \frac{3}{5} &= \int_{-\infty}^{\infty} x f(x)dx\\
             &= \int_{0}^{1} x (a+bx^2)dx\\
             &= \int_{0}^{1} ax+bx^3dx\\
             &= \frac{a}{2}x^2+\frac{b}{4}x^4 \Big|_0^1\\
             &= \frac{a}{2}1^2+\frac{b}{4}1^4 - (\frac{a}{2}0^2+\frac{b}{4}0^4 )\\
        \frac{3}{5} &= \frac{a}{2}+\frac{b}{4}
    \end{align*}
    \begin{align*}
        F(x) = 1 &= \int_{-\infty}^{\infty} f(x)dx\\
             &= \int_{0}^{1} (a+bx^2)dx\\
             &= ax + \frac{b}{3}x^3 \Big|_0^1\\
             &= a\cdot1 + \frac{b}{3}1^3 - (a\cdot0 + \frac{b}{3}0^3) \\
             1 &= a + \frac{b}{3}
    \end{align*}

    \begin{align*}
        \frac{3}{5} &= \frac{a}{2}+\frac{b}{4}\\
        \frac{3}{5} &= \frac{1 - \frac{b}{3}}{2}+\frac{b}{4}\\
        \frac{3}{5} &= \frac{3-b}{6}+\frac{b}{4}\\
        b           &= \frac{6}{5}\\
        a           &= \frac{3}{5}
    \end{align*}
    
\end{homeworkProblem}

\pagebreak

\begin{homeworkProblem}[30]
    Suppose that X has density function

    \begin{center} f(x) = \begin{cases}
        1 & 0 < x < 1\\
        0 & \text{otherwise}\\
    \end{cases} \end{center}

    Compute $E[X^n]$

    \begin{enumerate}[label=(\alph*)]
        \item by computing the density of $X^n$ and then using the definition of expectation

            \textbf{Solution}

            Let $Y$ be $X^n$. $\forall \, 0\leq y \leq 1$, let the cumulative distribution function of $Y$ be
            \begin{align*}
                F_Y(y) &= P\{Y \leq y\}\\
                       &= P\{X^n \leq y\}\\
                       &= P\{X \leq \sqrt[n]y\}\\
                       &= \int_0^{\sqrt[n]y} 1 \,dx\\
                       &= x \Big|_0^{\sqrt[n]y} \\
                F_Y(y) &= \sqrt[n]y \\
            \end{align*}
            Now we differentiate $F_Y(y)$ to get the density function of $Y$.
            \begin{align*}
                f_Y(y) &= \frac{d}{dy} F_Y(y) \\
                       &= \frac{d}{dy} \sqrt[n]y \\
                       &= \frac{y^{\frac{1}{n} - 1}}{n} \\
                f_Y(y) &= \frac{1}{n} y^{\frac{1-n}{n}}, 0 \leq y \leq 1\\
            \end{align*}

            Now we use the definition of expectation to find $E[Y]$.
            \begin{align*}
                E[Y] &= \int_{-\infty}^{\infty} y f_Y(y) \,dy\\
                     &= \int_0^1 y\left(\frac{1}{n} y^{\frac{1-n}{n}}\right) \,dy\\
                     &= \frac{1}{n} \int_0^1 y^{\frac{1-n}{n} + 1} \,dy\\
                     &= \frac{1}{n} \int_0^1 y^{\frac{1}{n}} \,dy\\
                     &= \frac{1}{n} \left(\frac{n}{1+n}\right) y^{\frac{1+n}{n}} \Big|_0^1 \\
                     &= \frac{1}{n+1} 1^{\frac{1+n}{n}} \\
                E[Y] = E[X^n] &= \frac{1}{n+1}\\
            \end{align*}
        \item by using Proposition 4.5.1
            
            If X is a continuous random variable with probability density function f(x), then for any real-valued function g,
            \[E[g(X)] = \int_{-\infty}^{\infty}g(x)f(x)dx\]

            \textbf{Solution}
            \begin{align*}
                E[g(X)] &= \int_{-\infty}^{\infty}g(x)f(x)dx \\
                E[X^n] &= \int_{-\infty}^{\infty}x^n f(x)dx \\
                       &= \int_0^1 x^n 1 \,dx \\
                       &= \frac{x^{n+1}}{n+1} \Big|_0^1 \\
                       &= \left(\frac{1^{n+1}}{n+1}\right) - 
                          \left(\frac{0^{n+1}}{n+1}\right) \\
                E[X^n] &= \frac{1}{n+1} \\
            \end{align*}

            
    \end{enumerate}
\end{homeworkProblem}

\pagebreak

\begin{homeworkProblem}[32]
    If $E[X] = 2$ and $E[X^2] = 8$, calculate
    \begin{enumerate}[label=(\alph*)]
        \item 
            \begin{align*}
                E[(2+4X)^2] &= E[4+ 16X + 16X^2] \\
                            &= E[4] + E[16X] + E[16X^2] \\
                            &= E[4] + 16E[X] + 16E[X^2] \\
                            &= 4 + 16\cdot 2 + 16\cdot 8 \\
                            &= 164
            \end{align*}
        \item $E[X^2 + (X+1)^2]$
            \begin{align*}
                E[X^2 + (X+1)^2] &= E[X^2 + X^2 + 2X + 1]\\
                                 &= E[2X^2 + 2X + 1]\\
                                 &= 2E[X^2] + 2E[X] + E[1]\\
                                 &= 2\cdot 8 + 2\cdot 2 + 1\\
                                 &= 21
            \end{align*}
    \end{enumerate}
\end{homeworkProblem}

\pagebreak

\begin{homeworkProblem}[34]
    If X is a continuous random variable having distribution function F, then its median is defined as that value for $m$ for which
    \[F(m) = \frac{1}{2}\]

    Find the median of the random variables with density function
    \begin{enumerate}[label=(\alph*)]
        \item $f(x) = e^{-x}, x \geq 0$
            \begin{align*}
                F(x) &= \int_{-\infty}^x f(x)dx\\
                     &= \int_0^x e^{-x} dx\\
                     &= -e^{-x} \Big|_0^x\\
                     &= (-e^{-x}) - (-e^0)\\
                F(x) = \frac{1}{2} &= 1 -e^{-x}\\
                \frac{1}{2} &= e^{-x}\\
                \ln(2^{-1}) &= \ln(e^{-x})\\
                \ln(2) &= x\\
            \end{align*}
        \item $f(x) = 1, 0 \leq x \leq 1$
            \begin{align*}
                F(x) &= \int_{-\infty}^x f(x)dx\\
                     &= \int_0^x 1 dx\\
                     &= x \Big|_0^x \\
                F(x) = \frac{1}{2} &= x\\
            \end{align*}
    \end{enumerate}
\end{homeworkProblem}

\pagebreak

\begin{homeworkProblem}[36]
    We say that $m_p$ is the $100p$ percentile of the distribution function F if

    \[F(m_p) = p\]

    Find $m_p$ for the distribution having density function

    \[f(x) = 2e^{-2x}, x \geq 0\]

    \textbf{Solution}

    \begin{align*}
        F(x) &= \int_{-\infty}^x f(x)dx\\
             &= \int_0^x 2e^{-2x} dx\\
             &= -e^{-2x} \Big|_0^x \\
             &= (-e^{-2x}) - (-e^{-2*0}) \\
             &= 1 -e^{-2x}\\
        F(x) &= 1 -e^{-2x}\\
        F(m_p) = p &= 1 -e^{-2m_p}\\
        1-p  &= e^{-2m_p}\\
        \ln(1-p) &= -2m_p \ln(e)\\
        \frac{-\ln(1-p)}{2} &= m_p\\
    \end{align*}
\end{homeworkProblem}

\pagebreak

\begin{homeworkProblem}[38]
    Compute the expectation and variance of the number of successes in $n$ independent trials, each of which results in a success with probability $p$. Is independence necessary?

    \textbf{Solution}

    Let random variable $X$ represent the number of successes in n independent trials.\\

    Each trial can be a success (1) or a failure (0). We can denote the success of an arbitrary $i^{th}$ trial with $X_i = 0 \text{ or } 1$.\\

    If each trial has a success chance of $p$ and there are $n$ trials, then the expectation of $X$ must be
    \[E[X] = \sum_{i = 1}^n p = np\]

    To calculate variance, we must first calculate the variance of one arbitrary trial $X_i$.
    \begin{align*}
        Var(X_i) &= E[X_i ^2] - (E[X_i])^2\\
                 &= E[(\text{0 or 1}) ^2] - (p)^2\\
                 &= E[X_i] - p^2\\
                 &= p - p^2\\
    \end{align*}

    The sum of the variances for each trial $X_i$ is the variance of $X$.

    \begin{align*}
        Var(X) &= \sum_{i=0}^n p-p^2\\
        Var(X) &= n(p-p^2)\\
    \end{align*}

    Independence is necessary for $\sum Var(X_i) = Var(X)$ to hold.
\end{homeworkProblem}

\pagebreak

\begin{homeworkProblem}[44]
    Let $X_i$ denote the percentage of votes cast in a given election that are for candidate $i$, and suppose that $X_1$ and $X_2$ have a joint density function

    \begin{center} $f_{X_1, X_2}(x, y)$ = \begin{cases}
        3(x+y), & \text{if } x\geq0, y\geq0, 0\leq x + y\leq 1\\
        0,      & \text{otherwise}
    \end{cases} \end{center}

    \begin{enumerate}[label=(\alph*)]
        \item Find the marginal densities of $X_1$ and $X_2$.

            \begin{align*}
                f_{X_1}(x) &= \int_{-\infty}^{\infty} f(x, y)dy\\
                           &= \int_0^{1-x} (3x+3y) dy\\
                           &= 3xy + \frac{3y^2}{2} \Big|_0^{1-x}\\
                           &= 3x(1-x) + \frac{3(1-x)^2}{2} - (3x(0) + \frac{3\cdot0^2}{2}) \\
                           &= 3(x-x^2) + \frac{3(1 -2x + x^2)}{2} \\
                           &= \frac{3(2x -2x^2 + 1 - 2x + x^2)}{2}\\
                f_{X_1}(x) &= \frac{3(1 - x^2)}{2}
            \end{align*}
            \begin{align*}
                f_{X_2}(x) &= \int_{-\infty}^{\infty} f(x, y)dx\\
                           &= \int_0^{1-x} (3x+3y) dx\\
                           &= 3yx + \frac{3x^2}{2} \Big|_0^{1-y}\\
                           &= 3y(1-y) + \frac{3(1-y)^2}{2} - (3y(0) + \frac{3\cdot0^2}{2}) \\
                           &= 3(y-y^2) + \frac{3(1 -2y + y^2)}{2} \\
                           &= \frac{3(2y -2y^2 + 1 - 2y + y^2)}{2}\\
                f_{X_2}(y) &= \frac{3(1 - y^2)}{2}
            \end{align*}
        \item Find $E[X_i]$ and Var$(X_i)$ for $i = 1, 2$.
            \begin{align*}
                E[X_i] &= \int_{-\infty}^{\infty} x f_{X_i}(x) dx\\
                       &= \int_0^{1} x \frac{3(1-x^2)}{2} dx\\
                       &= \int_0^{1} \frac{3(x-x^3)}{2} dx\\
                       &= \frac{3}{2} (\frac{x^2}{2} - \frac{x^4}{4}) \Big|_0^1\\
                       &= \frac{3}{2} \left(
                              (\frac{1^2}{2} - \frac{1^4}{4}) 
                              - (\frac{0^2}{2} - \frac{0^4}{4})
                          \right)\\
                       &= \frac{3}{2} (\frac{1}{2} - \frac{1}{4}) \\
                       &= \frac{3}{4} - \frac{3}{8} \\
                E[X_i] &= \frac{3}{8} \\
            \end{align*}
            \begin{align*}
                Var(X_i) &= \int_{-\infty}^{\infty} x^2 f_{X_i}(x) dx - E[X_i]^2\\
                         &= \int_0^{1} x^2 \frac{3(1-x^2)}{2} dx - \left(\frac{3}{8}\right)^2\\
                         &= \int_0^{1} \frac{3(x^2-x^4)}{2} dx - \frac{9}{64}\\
                         &= \frac{3}{2} (\frac{x^3}{3} - \frac{x^5}{5}) \Big|_0^1 - \frac{9}{64}\\
                         &= \frac{3}{2} \left(
                                (\frac{1^3}{3} - \frac{1^5}{5}) 
                                - (\frac{0^3}{3} - \frac{0^5}{5})
                            \right) - \frac{9}{64}\\
                         &= \frac{3}{2} \left(
                                (\frac{1}{3} - \frac{1}{5}) 
                            \right) - \frac{9}{64}\\
                         &= \frac{1}{5} - \frac{9}{64}\\
                Var(X_i) &= \frac{19}{320}
            \end{align*}
    \end{enumerate}
\end{homeworkProblem}

\pagebreak

\begin{homeworkProblem}
    A product is classified according to the number of defects it contains and the factory that produces it. Let $X_1$ and $X_2$ be the random variables that represent the number of defects per unit (taking on possible values of 0, 1, 2, or 3) and the factory number (taking on possible values 1 or 2), respectively. The entries in the table represent the joint possibility mass function of a randomly chosen product.

    \begin{center}
    \begin{tabular}{c | c c}
        \diagbox{$X_1$}{$X_2$} & 1 & 2\\\hline
         0 & $\frac{1}{8} $ & $\frac{1}{16}$ \\
         1 & $\frac{1}{16}$ & $\frac{1}{16}$ \\
         2 & $\frac{3}{16}$ & $\frac{1}{8} $ \\
         3 & $\frac{1}{8} $ & $\frac{1}{4} $ \\
    \end{tabular}
    \end{center}

    \begin{enumerate}[label=(\alph*)]
        \item Find the marginal probability distributions of $X_1$ and $X_2$.
            \begin{center} $P_{X_1}(x)$ \begin{cases}
                    \frac{3}{16} & \text{if } x = 0\\
                    \frac{2}{16} & \text{if } x = 1\\
                    \frac{5}{16} & \text{if } x = 2\\
                    \frac{6}{16} & \text{if } x = 3\\
            \end{cases} \end{center}

            \[P_{X_2}(x) = \frac{1}{2}\]
        \item Find $E[X_1]$, $E[X_2]$, Var$(X_1)$, Var$(X_2)$, and Cov$(X_1, X_2)$.
            \begin{align*}
                E[X_1] &= (0)\frac{3}{16} + 
                         (1)\frac{2}{16} + 
                         (2)\frac{5}{16} + 
                         (3)\frac{6}{16} 
                         = \frac{30}{16}\\
                E[X_2] &= (1)\frac{1}{2} + (2)\frac{1}{2} = \frac{3}{2}
            \end{align*}
            \begin{align*}
                Var(X_1) &= \sum_{x = 0}^3 P_{X_1}x^2 - \left(E[X_1]\right)^2\\
                         &= (0)^2\frac{3}{16} +
                            (1)^2\frac{2}{16} +
                            (2)^2\frac{5}{16} +
                            (3)^2\frac{6}{16} 
                            - (\frac{30}{16})^2\\
                            &= \frac{76}{16} - (\frac{30}{16})^2 = \frac{79}{64}
            \end{align*}
            \begin{align*}
                Var(X_2) &= \sum_{x = 1}^2 P_{X_2}x^2 - \left(E[X_2]\right)^2\\
                         &= (1)^2\frac{1}{2} +
                            (2)^2\frac{1}{2}
                            - (\frac{3}{2})^2\\
                            &= \frac{5}{2} - \frac{9}{4} = \frac{1}{4}
            \end{align*}
            \begin{align*}
                cov(X_1, X_2) &= \sum_{0\leq x\leq 3, 1\leq y\leq 2} (x - E[X_1])(y - E[X_2])P(x, y)\\
                              &= (0 - \frac{30}{16})(1 - \frac{3}{2})(\frac{1}{8}) + 
                               (1 - \frac{30}{16})(1 - \frac{3}{2})(\frac{1}{16}) + 
                               (2 - \frac{30}{16})(1 - \frac{3}{2})(\frac{3}{16}) + 
                               (3 - \frac{30}{16})(1 - \frac{3}{2})(\frac{1}{8}) + \\
                              & (0 - \frac{30}{16})(2 - \frac{3}{2})(\frac{1}{16}) + 
                               (1 - \frac{30}{16})(2 - \frac{3}{2})(\frac{1}{16}) + 
                               (2 - \frac{30}{16})(2 - \frac{3}{2})(\frac{1}{8}) + 
                               (3 - \frac{30}{16})(2 - \frac{3}{2})(\frac{1}{4}) \\
            \end{align*}
            
    \end{enumerate}
\end{homeworkProblem}

\pagebreak

\begin{homeworkProblem}
    Find Corr$(X_1, X_2)$ for the random variables of Problem 44.

    \begin{center} $f_{X_1, X_2}(x, y)$ = \begin{cases}
        3(x+y), & \text{if } x\geq0, y\geq0, 0\leq x + y\leq 1\\
        0,      & \text{otherwise}
    \end{cases} \end{center}
    \[E[X_i] &= \frac{3}{8}\]
    \[Var(X_i) &= \frac{19}{320}\]

    \textbf{Solution}

    \[
        Corr(X_1, X_2) 
        = \frac{cov(X_1, X_2)}{\sigma_{X_1}\sigma_{X_2}}
        = \frac{E[X_1X_2] - E[X_1]E[X_2]}{\sqrt{Var(X_1)}\sqrt{Var(X_2)}}
    \]
    We must calculate the joint expectation of $X_1$ and $X_2$ to finish our calculation for the correlation!

    \begin{align*}
        E[X_1X_2] &= \int \int xy f_{X_1, X_2}(x, y) \,dx \,dy\\
                  &= \int_0^1 \int_0^{1-y} (xy) 3(x+y) \,dx \,dy\\
                  &= 3 \int_0^1 y \int_0^{1-y} x^2 + xy \,dx \,dy\\
                  &= 3 \int_0^1 y \left(\frac{x^3}{3} + \frac{x^2y}{2} \Big|_0^{1-y}\right) \,dy\\
                  &= 3 \int_0^1 y \left(
                      \left(\frac{(1-y)^3}{3} + \frac{(1-y)^2y}{2} \right) - 
                      \left(\frac{0^3}{3} + \frac{0^2y}{2} \right)
                  \right) \,dy\\
                  &= 3 \int_0^1 y
                      \left(\frac{(1-y)^3}{3} + \frac{(1-y)^2y}{2} \right)
                  \,dy\\
                  &= 3 \int_0^1 y
                      \left(\frac{-y^3+3y^2-3y+1}{3} 
                      + \frac{(y^2-2y+1)y}{2} \right)
                  \,dy\\
                  &= 3 \int_0^1
                      \frac{-y^4+3y^3-3y^2+y}{3} 
                      + \frac{y^4-2y^3+y^2}{2}
                  \,dy\\
                  &= \int_0^1
                      -\frac{2y^4}{2}
                      +\frac{6y^3}{2}
                      -\frac{6y^2}{2}
                      +\frac{2y}{2}
                      +\frac{3y^4}{2}
                      -\frac{6y^3}{2}
                      +\frac{3y^2}{2}
                  \,dy\\
                  &= \frac{1}{2}\int_0^1
                       y^4
                      -3y^2
                      +2y
                  \,dy\\
                  &= \frac{1}{2}\left(
                  \frac{y^5}{5} -y^3 +y^2
                  \right) \Big|_0^1\\
                  &= \frac{1}{2}
                  \left( \frac{1^5}{5} -1^3 +1^2 \right) - \left( \frac{0^5}{5} -0^3 +0^2 \right) \\
                  &= \frac{1}{2} \left( \frac{1}{5} -1 +1 \right) \\
                  &= \frac{1}{10} -\frac{1}{2} +\frac{1}{2} \\
         E[X_1X_2] &= \frac{1}{10}
    \end{align*}

    Now we will solve for the correlation.
    \begin{align*}
        Corr(X_1, X_2) &= \frac{E[X_1X_2] - E[X_1]E[X_2]}{\sqrt{Var(X_1)}\sqrt{Var(X_2)}}\\
                       &= \frac{\frac{1}{10} - \frac{3}{8}\frac{3}{8}}{\sqrt{\frac{19}{320}}\sqrt{\frac{19}{320}}}\\
                       &= \frac{\frac{1}{10} - \frac{9}{64}}{\frac{19}{320}}\\
        Corr(X_1, X_2) &= -\frac{13}{19} = -0.68\dots
    \end{align*}

\end{homeworkProblem}

\begin{homeworkProblem}[52]
    If $X_1$ and $X_2$ have the same probability distribution function, show that

    \[\text{Cov}(X_1 - X_2, X_1 + X_2) = 0\]

    Note that independence is not being assumed.\\

    \textbf{Solution}
    \begin{align*}
        Cov(A, B) &= E[AB] - E[A]E[B]\\
        Cov(X_1 - X_2, X_1 + X_2) &= E[X_1 - X_2X_1 + X_2] - E[X_1 - X_2]E[X_1 + X_2]\\
                                  &= E[X_1 - X_2X_1 + X_2] - (E[X_1] - E[X_2])(E[X_1] + E[X_2])\\
                                  &= E[0\,(X_1 + X_2)] - (0)(0)\\
                                  &= 0
    \end{align*}
\end{homeworkProblem}

\pagebreak

\begin{homeworkProblem}
    Suppose that X has density function
    \[f(x) = e^{-x}, x>0\]
    Compute the moment generating function of X and use your result to determine its mean and variance. Check your answer for the mean by a direct calculation.

    \textbf{Solution}
    \begin{align*}
        M_x(t) &= \int_{-\infty}^{\infty} e^{tx} f(x) \,dx\\
        &= \int_0^{\infty} e^{tx}e^{-x} \,dx\\
        &= \int_0^{\infty} e^{tx-x} \,dx\\
        &= \int_0^{\infty} e^{x(t-1)} \,dx\\
        &= \frac{e^{x(t-1)}}{t-1} \Big|_0^{\infty} \\
        &= \left(\frac{e^{\infty(t-1)}}{t-1}\right) - \left(\frac{e^{0(t-1)}}{t-1}\right)\\
        M_x(t) &= \frac{-1}{t-1}\\
    \end{align*}
    Calculating the mean
    \begin{align*}
        \mu &= \frac{d}{dt} M_x(t = 0) \\
            &= \frac{d}{dt} -(t-1)^{-1}\\
            &= (t-1)^{-2} \cdot 1\\
            &= (t-1)^{-2} \\
            &= (0-1)^{-2} \\
        \mu &= 1
    \end{align*}
    Calculating the variance
    \begin{align*}
        E[X^2] &= \frac{d^2}{dt} M_x(t = 0) \\
               &= \frac{d^2}{dt} -(t-1)^{-1}\\
               &= -2(t-1)^{-3} \\
               &= -2(0-1)^{-3} \\
        E[X^2] &= 2
    \end{align*}
    \begin{align*}
        \sigma^2 &= E[X^2] - (E[X])^2\\
                 &= 2 - (1)^2\\
        \sigma^2 &= 1
    \end{align*}
\end{homeworkProblem}

\begin{homeworkProblem}[57]
    Let $X$ and $Y$ have respective distribution functions $F_X$ and $F_Y$ , and suppose that for some constants $a$ and $b > 0$,
    
    \[F_X(x) = F_Y(\frac{x-a}{b})\]

    Hint: X has the same distribution as what other random variable?\\

    \begin{enumerate}[label=(\alph*)]
        \item Determine $E[X]$ in terms of $E[Y]$.

            \begin{align*}
                F_X(x) &= F_Y(\frac{x-a}{b})\\
                P(X\leq x) &= P(Y \leq \frac{x-a}{b})\\
                P(X\leq x) &= P(bY \leq x-a)\\
                P(X\leq x) &= P(bY + a \leq x)\\
                X &= bY + a\\
                E[X] &= E[bY + a]
            \end{align*}

        \item Determine Var(X) in terms of Var(Y).
            \[Var(X) &= Var(bY + a)\]
    \end{enumerate}

\end{homeworkProblem}
\end{document}
