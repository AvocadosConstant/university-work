\documentclass{article}

\usepackage{fancyhdr}
\usepackage{extramarks}
\usepackage{amsmath}
\usepackage{amsthm}
\usepackage{amsfonts}
\usepackage{tikz}
\usepackage{enumerate}
\usepackage{mathtools}
\usepackage{enumitem}
\usepackage{pgfplots}
\usepackage{diagbox}

%
% Basic Document Settings
%

\topmargin=-0.45in
\evensidemargin=0in
\oddsidemargin=0in
\textwidth=6.5in
\textheight=9.0in
\headsep=0.25in

\linespread{1.1}

\pagestyle{fancy}
\lhead{\hmwkAuthorName}
\chead{\hmwkClass\ (\hmwkClassInstructor\ \hmwkClassTime): \hmwkTitle}
\rhead{\firstxmark}
\lfoot{\lastxmark}
\cfoot{\thepage}

\renewcommand\headrulewidth{0.4pt}
\renewcommand\footrulewidth{0.4pt}

\setlength\parindent{0pt}

%
% Create Problem Sections
%

\newcommand{\enterProblemHeader}[1]{
    \nobreak\extramarks{}{Problem \arabic{#1} continued on next page\ldots}\nobreak{}
    \nobreak\extramarks{Problem \arabic{#1} (continued)}{Problem \arabic{#1} continued on next page\ldots}\nobreak{}
}

\newcommand{\exitProblemHeader}[1]{
    \nobreak\extramarks{Problem \arabic{#1} (continued)}{Problem \arabic{#1} continued on next page\ldots}\nobreak{}
    \stepcounter{#1}
    \nobreak\extramarks{Problem \arabic{#1}}{}\nobreak{}
}

\setcounter{secnumdepth}{0}
\newcounter{partCounter}
\newcounter{homeworkProblemCounter}
\setcounter{homeworkProblemCounter}{1}
\nobreak\extramarks{Problem \arabic{homeworkProblemCounter}}{}\nobreak{}

%
% Homework Problem Environment
%
% This environment takes an optional argument. When given, it will adjust the
% problem counter. This is useful for when the problems given for your
% assignment aren't sequential. See the last 3 problems of this template for an
% example.
%
\newenvironment{homeworkProblem}[1][-1]{
    \ifnum#1>0
    \setcounter{homeworkProblemCounter}{#1}
    \fi
    \section{Problem \arabic{homeworkProblemCounter}}
    \setcounter{partCounter}{1}
    \enterProblemHeader{homeworkProblemCounter}
    }{
    \exitProblemHeader{homeworkProblemCounter}
}

%
% Homework Details
%   - Title
%   - Due date
%   - Class
%   - Section/Time
%   - Instructor
%   - Author
%

% TODO: Replace with correct number and date
\newcommand{\hmwkTitle}{Problem Set\ \#10}
\newcommand{\hmwkDueDate}{May 08, 2017}
\newcommand{\hmwkClass}{MATH 327}
\newcommand{\hmwkClassInstructor}{Professor Mei-Hsiu Chen}
\newcommand{\hmwkAuthorName}{Tim Hung}

%
% Title Page
%

\title{
    \vspace{2in}
    \textmd{\textbf{\hmwkClass:\ \hmwkTitle}}\\
    \normalsize\vspace{0.1in}\small{Due\ on\ \hmwkDueDate\ at 2:10pm}\\
    \vspace{0.1in}\large{\textit{\hmwkClassInstructor\ \hmwkClassTime}}\\
}

\author{\textbf{\hmwkAuthorName}}
\date{}

\renewcommand{\part}[1]{\textbf{\large Part \Alph{partCounter}}\stepcounter{partCounter}\\}

%
% Various Helper Commands
%

% Useful for algorithms
\newcommand{\alg}[1]{\textsc{\bfseries \footnotesize #1}}

% For derivatives
\newcommand{\deriv}[1]{\frac{\mathrm{d}}{\mathrm{d}x} (#1)}

% For partial derivatives
\newcommand{\pderiv}[2]{\frac{\partial}{\partial #1} (#2)}

% Integral dx
\newcommand{\dx}{\mathrm{d}x}

% Alias for the Solution section header
\newcommand{\solution}{\textbf{\large Solution}}

% Probability commands: Expectation, Variance, Covariance, Bias
\newcommand{\E}{\mathrm{E}}
\newcommand{\Var}{\mathrm{Var}}
\newcommand{\Cov}{\mathrm{Cov}}
\newcommand{\Bias}{\mathrm{Bias}}

\begin{document}

\maketitle

\pagebreak

\begin{homeworkProblem}[27]

    A sample of 10 fish were caught at lake A and their PCB concentrations were
    measured using a certain technique. The resulting data in parts per million were

    \[ \text{Lake A:  } 11.5, 10.8, 11.6, 9.4, 12.4, 11.4, 12.2, 11, 10.6, 10.8 \]

    In addition, a sample of 8 fish were caught at lake B and their levels of PCB were
    measured by a different technique than that used at lake A. The resultant data
    were

    \[ \text{Lake B:  } 11.8, 12.6, 12.2, 12.5, 11.7, 12.1, 10.4, 12.6 \]

    If it is known that the measuring technique used at lake A has a variance of .09
    whereas the one used at lake B has a variance of .16, could you reject (at the
    5 percent level of significance) a claim that the two lakes are equally contaminated?\\

    \textbf{Solution} 

    Yes, you can reject a claim that the two lakes are equally contaminated.

    Calculated $p\text{-value} \approx 0$
    
\end{homeworkProblem}

\begin{homeworkProblem}[28]

    A method for measuring the pH level of a solution yields a measurement value
    that is normally distributed with a mean equal to the actual pH of the solution
    and with a standard deviation equal to .05. An environmental pollution scientist
    claims that two different solutions come from the same source. If this were so,
    then the pH level of the solutions would be equal. To test the plausibility of
    this claim, 10 independent measurements were made of the pH level for both
    solutions, with the following data resulting.

    \begin{center}
        TABLE GOES HERE
    \end{center}

    \begin{enumerate}[label=(\alph*)]
        \item Do the data disprove the scientist’s claim? Use the 5 percent level of significance.

            No

        \item What is the p-value?

            \[ p\text{-value} = P(Z > 0.81) = .420 \]

    \end{enumerate}
    
\end{homeworkProblem}

\pagebreak

\begin{homeworkProblem}[29]

    The following are the values of independent samples from two different
    populations.

    \begin{center}
        TABLE GOES HERE
    \end{center}

    Let $\mu_1$ and $\mu_2$ be the respective means of the two populations. Find the p-value
    of the test of the null hypothesis
    \[ H0 : \mu_1 \leq \mu_2 \]
    versus the alternative
    \[ H1 : \mu_1 > \mu_2 \]
    when the population standard deviations are $\sigma_1 = 10$ and

    \begin{enumerate}[label=(\alph*)]
        \item $\sigma_2 = 5$

            \[ p\text{-value} = 0.004 \]

        \item $\sigma_2 = 10$

            \[ p\text{-value} = 0.018 \]

        \item $\sigma_2 = 20$

            \[ p\text{-value} = 0.092 \]

    \end{enumerate}
    
\end{homeworkProblem}

\begin{homeworkProblem}[31]

    The viscosity of two different brands of car oil is measured and the following data
    resulted:

    \begin{center}
        TABLE GOES HERE
    \end{center}

    Test the hypothesis that the mean viscosity of the two brands is equal, assuming
    that the populations have normal distributions with equal variances.\\

    \textbf{Solution} 
    
    \[ p\text{-value} = 2P(T_11 > 1.75) = .420 \]

\end{homeworkProblem}

\pagebreak

\begin{homeworkProblem}[33]

    Twenty-five men between the ages of 25 and 30, who were participating in a wellknown
    heart study carried out in Framingham, Massachusetts, were randomly
    selected. Of these, 11 were smokers and 14 were not. The following data refer to
    readings of their systolic blood pressure.

    \begin{center}
        TABLE GOES HERE
    \end{center}

    Use these data to test the hypothesis that the mean blood pressures of smokers
    and nonsmokers are the same.\\

    \textbf{Solution} 
    
    \[ p\text{-value} = 0.019 \]

\end{homeworkProblem}

\begin{homeworkProblem}[35]

    A professor claims that the average starting salary of industrial engineering graduating
    seniors is greater than that of civil engineering graduates. To study this
    claim, samples of 16 industrial engineers and 16 civil engineers, all of whom graduated
    in 2006, were chosen and sample members were queried about their starting
    salaries. If the industrial engineers had a sample mean salary of \$72,700 and a
    sample standard deviation of \$2,400, and the civil engineers had a sample mean
    salary of \$71,400 and a sample standard deviation of \$2,200, has the professor’s
    claim been verified? Find the appropriate p-value.\\

    \textbf{Solution} 

    \[ p\text{-value} = P(T_{30} > 1.60) = .06 \]
    
    Claim is unverified. Not at 5\% significance.
    
\end{homeworkProblem}

\begin{homeworkProblem}[38]

    To learn about the feeding habits of bats, 22 bats were tagged and tracked by
    radio. Of these 22 bats, 12 were female and 10 were male. The distances flown
    (in meters) between feedings were noted for each of the 22 bats, and the following
    summary statistics were obtained.

    \begin{center}
        TABLE GOES HERE
    \end{center}

    Test the hypothesis that the mean distance flown between feedings is the same
    for the populations of both male and of female bats. Use the 5 percent level of
    significance.\\

    \textbf{Solution} 
    
    Cannot reject null hypothesis

    Test statistics = 1.15

\end{homeworkProblem}

\pagebreak

\begin{homeworkProblem}[42]

    Ten pregnant women were given an injection of pitocin to induce labor. Their
    systolic blood pressures immediately before and after the injection were:

    \begin{center}
        TABLE GOES HERE
    \end{center}

    Do the data indicate that injection of this drug changes blood pressure?\\

    \textbf{Solution} 
    
    \[ p\text{-value} = 2P(T_9 > 2.33) = .044 \]

    Data does not indicate the injection changes blood pressure: Rejected at 5\% level of significance.

\end{homeworkProblem}

\begin{homeworkProblem}[47]

    A pharmaceutical house produces a certain drug item whose weight has a standard
    deviation of .5 milligrams. The company’s research team has proposed
    a new method of producing the drug. However, this entails some costs and
    will be adopted only if there is strong evidence that the standard deviation of
    the weight of the items will drop to below .4 milligrams. If a sample of 10
    items is produced and has the following weights, should the new method be
    adopted?

    \begin{center}
        TABLE GOES HERE
    \end{center}

    \textbf{Solution} 

    Test H0 : $\sigma \geq .4$

    \[ \frac{9S^2}{0.4^2} &= 9.252 \times 10^4 \]       

    \[p\text{-value} &= P(X^2_9 < .000925) < .0001 \]
    
    Null hypothesis is rejected. 
    
    New method should be adopted.
    
\end{homeworkProblem}

\begin{homeworkProblem}[48]

    The production of large electrical transformers and capacitators requires the use of
    polychlorinated biphenyls (PCBs), which are extremely hazardous when released
    into the environment. Two methods have been suggested to monitor the levels
    of PCB in fish near a large plant. It is believed that each method will result in
    a normal random variable that depends on the method. Test the hypothesis at
    the $\alpha = .10$ level of significance that both methods have the same variance, if a
    given fish is checked 8 times by each method with the following data (in parts per
    million) recorded.

    \begin{center}
        TABLE GOES HERE
    \end{center}

    \textbf{Solution} 

    \[ \frac{S_1^2}{S_2^2} = .5317 \]
    \[ p\text{-value} = 2P(F_{7.7} < .5317) = .42 \]
    
    Hypothesis is accepted.
    
\end{homeworkProblem}

\pagebreak

\begin{homeworkProblem}[49]

    In Problem 31, test the hypothesis that the populations have the same variances.\\

    \textbf{Solution} 

    \[ \frac{S_1^2}{S_2^2} = 14.05 \]
    \[ p\text{-value} = 2P(F_{5.6} > 14.05) = 0.006 \]
    
    Hypothesis is rejected.

\end{homeworkProblem}

\begin{homeworkProblem}[58]

    A standard drug is known to be effective in 75 percent of the cases in which it
    is used to treat a certain infection. A new drug has been developed and has been
    found to be effective in 42 cases out of 50. Based on this, would you accept, at
    the 5 percent level of significance, the hypothesis that the two drugs are of equal
    effectiveness? What is the p-value?\\

    \textbf{Solution} 

    \[ p\text{-value} = 2P(Bin(50, .75) \geq 42) = .183 \]
    
\end{homeworkProblem}

\begin{homeworkProblem}[59]

    Do Problem 58 by using a test based on the normal approximation to the
    binomial.\\

    \textbf{Solution} 

    \[ p\text{-value} = 2P\left(Z > \frac{41.5 - \frac{150}{4}}{\sqrt{\frac{150}{16}}}\right) \geq 42) = .183 \]

\end{homeworkProblem}

\end{document}
