\documentclass{article}

\usepackage{fancyhdr}
\usepackage{extramarks}
\usepackage{amsmath}
\usepackage{amsthm}
\usepackage{amsfonts}
\usepackage{tikz}
\usepackage[plain]{algorithm}
\usepackage{algpseudocode}
\usepackage{enumerate}
\usepackage{mathtools}
\usepackage{enumitem}

\usetikzlibrary{automata,positioning}

%
% Basic Document Settings
%

\topmargin=-0.45in
\evensidemargin=0in
\oddsidemargin=0in
\textwidth=6.5in
\textheight=9.0in
\headsep=0.25in

\linespread{1.1}

\pagestyle{fancy}
\lhead{\hmwkAuthorName}
\chead{\hmwkClass\ (\hmwkClassInstructor\ \hmwkClassTime): \hmwkTitle}
\rhead{\firstxmark}
\lfoot{\lastxmark}
\cfoot{\thepage}

\renewcommand\headrulewidth{0.4pt}
\renewcommand\footrulewidth{0.4pt}

\setlength\parindent{0pt}

%
% Create Problem Sections
%

\newcommand{\enterProblemHeader}[1]{
    \nobreak\extramarks{}{Problem \arabic{#1} continued on next page\ldots}\nobreak{}
    \nobreak\extramarks{Problem \arabic{#1} (continued)}{Problem \arabic{#1} continued on next page\ldots}\nobreak{}
}

\newcommand{\exitProblemHeader}[1]{
    \nobreak\extramarks{Problem \arabic{#1} (continued)}{Problem \arabic{#1} continued on next page\ldots}\nobreak{}
    \stepcounter{#1}
    \nobreak\extramarks{Problem \arabic{#1}}{}\nobreak{}
}

\setcounter{secnumdepth}{0}
\newcounter{partCounter}
\newcounter{homeworkProblemCounter}
\setcounter{homeworkProblemCounter}{1}
\nobreak\extramarks{Problem \arabic{homeworkProblemCounter}}{}\nobreak{}

%
% Homework Problem Environment
%
% This environment takes an optional argument. When given, it will adjust the
% problem counter. This is useful for when the problems given for your
% assignment aren't sequential. See the last 3 problems of this template for an
% example.
%
\newenvironment{homeworkProblem}[1][-1]{
    \ifnum#1>0
    \setcounter{homeworkProblemCounter}{#1}
    \fi
    \section{Problem \arabic{homeworkProblemCounter}}
    \setcounter{partCounter}{1}
    \enterProblemHeader{homeworkProblemCounter}
    }{
    \exitProblemHeader{homeworkProblemCounter}
}

%
% Homework Details
%   - Title
%   - Due date
%   - Class
%   - Section/Time
%   - Instructor
%   - Author
%

\newcommand{\hmwkTitle}{Problem Set\ \#3}
\newcommand{\hmwkDueDate}{February 13, 2017}
\newcommand{\hmwkClass}{MATH 327}
\newcommand{\hmwkClassInstructor}{Professor Mei-Hsiu Chen}
\newcommand{\hmwkAuthorName}{Tim Hung}

%
% Title Page
%

\title{
    \vspace{2in}
    \textmd{\textbf{\hmwkClass:\ \hmwkTitle}}\\
    \normalsize\vspace{0.1in}\small{Due\ on\ \hmwkDueDate\ at 1:10pm}\\
    \vspace{0.1in}\large{\textit{\hmwkClassInstructor\ \hmwkClassTime}}\\
}

\author{\textbf{\hmwkAuthorName}}
\date{}

\renewcommand{\part}[1]{\textbf{\large Part \Alph{partCounter}}\stepcounter{partCounter}\\}

%
% Various Helper Commands
%

% Useful for algorithms
\newcommand{\alg}[1]{\textsc{\bfseries \footnotesize #1}}

% For derivatives
\newcommand{\deriv}[1]{\frac{\mathrm{d}}{\mathrm{d}x} (#1)}

% For partial derivatives
\newcommand{\pderiv}[2]{\frac{\partial}{\partial #1} (#2)}

% Integral dx
\newcommand{\dx}{\mathrm{d}x}

% Alias for the Solution section header
\newcommand{\solution}{\textbf{\large Solution}}

% Probability commands: Expectation, Variance, Covariance, Bias
\newcommand{\E}{\mathrm{E}}
\newcommand{\Var}{\mathrm{Var}}
\newcommand{\Cov}{\mathrm{Cov}}
\newcommand{\Bias}{\mathrm{Bias}}

\begin{document}

\maketitle

\pagebreak

\begin{homeworkProblem}
    Five men and 5 women are ranked according to their scores on an examination.  Assume that no two scores are alike and all 10! possible rankings are equally likely. Let X denote the highest ranking achieved by a woman (for instance, $X = 2$ if the top-ranked person was male and the next-ranked person was female). Find $P\{X = i\}, i = 1, 2, 3, . . . , 8, 9, 10$.
\end{homeworkProblem}

\begin{homeworkProblem}
    Let X represent the difference between the number of heads and the number of tails obtained when a coin is tossed n times. What are the possible values of X ?
\end{homeworkProblem}

\begin{homeworkProblem}
    In Problem 2, if the coin is assumed fair, for n = 3, what are the probabilities
    associated with the values that X can take on?
\end{homeworkProblem}

\begin{homeworkProblem}
    The distribution function of the random variable X is given

    \[
        F(x) = 
        \begin{cases}
            0 & x < 0\\        
            \frac{x}{2} & 0 \leq x < 1\\
            \frac{2}{3} & 1 \leq x < 2\\
            \frac{11}{12} & 2 \leq x < 3\\
            1 & 3 \leq x
        \end{cases}
    \]

    \begin{enumerate}[label=(\alph*)]
        \item Plot this distribution function.
        \item What is $P\{X > \frac{1}{2} \}$?
        \item What is $P\{2 < X \leq 4\}$?
        \item What is $P\{X < 3\}$?
        \item What is $P\{X = 1\}$?
    \end{enumerate}
\end{homeworkProblem}

\begin{homeworkProblem}
    Suppose the random variable X has probability density function

    \[
        f(x) =
        \begin{cases}
            cx^3, & \text{if } 0\leq x \leq 1\\
            0, & \text{otherwise}\\
        \end{cases}
    \]
    \begin{enumerate}[label=(\alph*)]
        \item Find the value of c.
        \item Find $P\{.4 < X < .8\}$.
    \end{enumerate}
\end{homeworkProblem}

\begin{homeworkProblem}[9]
    A set of five transistors are to be tested, one at a time in a random order, to see which of them are defective. Suppose that three of the five transistors are defective, and let $N_1$ denote the number of tests made until the first defective is spotted, and let $N_2$ denote the number of additional tests until the second defective is spotted.  Find the joint probability mass function of $N_1$ and $N_2$.
\end{homeworkProblem}

\begin{homeworkProblem}
    The joint probability density function of X and Y is given by
    \[
        f(x,y) = \frac{6}{7}(x^2 + \frac{xy}{2}), 0 < x < 1, 0 < y < 2
    \]
    \begin{enumerate}[label=(\alph*)]
        \item Verify that this is indeed a joint density function.
        \item Compute the density function of X.
        \item Find $P\{X>Y\}$.
    \end{enumerate}
\end{homeworkProblem}


\begin{homeworkProblem}[12]
    The joint density of X and Y is given by
    \[
        f(x, y) =
        \begin{cases}
            xe^{-(x+y)}, & x > 0, y > 0\\
            0, & \text{otherwise}\\
        \end{cases}
    \]
    \begin{enumerate}[label=(\alph*)]
        \item Compute the denisty of X.
        \item Compute the density of Y.
        \item Are X and Y independent?
    \end{enumerate}
\end{homeworkProblem}

\begin{homeworkProblem}
    The joint density of X and Y is given by
    \[
        f(x, y) =
        \begin{cases}
            2, & 0 < x < y, 0 < y < 1\\
            0, & \text{otherwise}\\
        \end{cases}
    \]
    \begin{enumerate}[label=(\alph*)]
        \item Compute the denisty of X.
        \item Compute the density of Y.
        \item Are X and Y independent?
    \end{enumerate}
\end{homeworkProblem}

\end{document}
