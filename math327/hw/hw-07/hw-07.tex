\documentclass{article}

\usepackage{fancyhdr}
\usepackage{extramarks}
\usepackage{amsmath}
\usepackage{amsthm}
\usepackage{amsfonts}
\usepackage{tikz}
\usepackage{enumerate}
\usepackage{mathtools}
\usepackage{enumitem}
\usepackage{pgfplots}
\usepackage{diagbox}

%
% Basic Document Settings
%

\topmargin=-0.45in
\evensidemargin=0in
\oddsidemargin=0in
\textwidth=6.5in
\textheight=9.0in
\headsep=0.25in

\linespread{1.1}

\pagestyle{fancy}
\lhead{\hmwkAuthorName}
\chead{\hmwkClass\ (\hmwkClassInstructor\ \hmwkClassTime): \hmwkTitle}
\rhead{\firstxmark}
\lfoot{\lastxmark}
\cfoot{\thepage}

\renewcommand\headrulewidth{0.4pt}
\renewcommand\footrulewidth{0.4pt}

\setlength\parindent{0pt}

%
% Create Problem Sections
%

\newcommand{\enterProblemHeader}[1]{
    \nobreak\extramarks{}{Problem \arabic{#1} continued on next page\ldots}\nobreak{}
    \nobreak\extramarks{Problem \arabic{#1} (continued)}{Problem \arabic{#1} continued on next page\ldots}\nobreak{}
}

\newcommand{\exitProblemHeader}[1]{
    \nobreak\extramarks{Problem \arabic{#1} (continued)}{Problem \arabic{#1} continued on next page\ldots}\nobreak{}
    \stepcounter{#1}
    \nobreak\extramarks{Problem \arabic{#1}}{}\nobreak{}
}

\setcounter{secnumdepth}{0}
\newcounter{partCounter}
\newcounter{homeworkProblemCounter}
\setcounter{homeworkProblemCounter}{1}
\nobreak\extramarks{Problem \arabic{homeworkProblemCounter}}{}\nobreak{}

%
% Homework Problem Environment
%
% This environment takes an optional argument. When given, it will adjust the
% problem counter. This is useful for when the problems given for your
% assignment aren't sequential. See the last 3 problems of this template for an
% example.
%
\newenvironment{homeworkProblem}[1][-1]{
    \ifnum#1>0
    \setcounter{homeworkProblemCounter}{#1}
    \fi
    \section{Problem \arabic{homeworkProblemCounter}}
    \setcounter{partCounter}{1}
    \enterProblemHeader{homeworkProblemCounter}
    }{
    \exitProblemHeader{homeworkProblemCounter}
}

%
% Homework Details
%   - Title
%   - Due date
%   - Class
%   - Section/Time
%   - Instructor
%   - Author
%

\newcommand{\hmwkTitle}{Problem Set\ \#7}
\newcommand{\hmwkDueDate}{March 29, 2017}
\newcommand{\hmwkClass}{MATH 327}
\newcommand{\hmwkClassInstructor}{Professor Mei-Hsiu Chen}
\newcommand{\hmwkAuthorName}{Tim Hung}

%
% Title Page
%

\title{
    \vspace{2in}
    \textmd{\textbf{\hmwkClass:\ \hmwkTitle}}\\
    \normalsize\vspace{0.1in}\small{Due\ on\ \hmwkDueDate\ at 2:10pm}\\
    \vspace{0.1in}\large{\textit{\hmwkClassInstructor\ \hmwkClassTime}}\\
}

\author{\textbf{\hmwkAuthorName}}
\date{}

\renewcommand{\part}[1]{\textbf{\large Part \Alph{partCounter}}\stepcounter{partCounter}\\}

%
% Various Helper Commands
%

% Useful for algorithms
\newcommand{\alg}[1]{\textsc{\bfseries \footnotesize #1}}

% For derivatives
\newcommand{\deriv}[1]{\frac{\mathrm{d}}{\mathrm{d}x} (#1)}

% For partial derivatives
\newcommand{\pderiv}[2]{\frac{\partial}{\partial #1} (#2)}

% Integral dx
\newcommand{\dx}{\mathrm{d}x}

% Alias for the Solution section header
\newcommand{\solution}{\textbf{\large Solution}}

% Probability commands: Expectation, Variance, Covariance, Bias
\newcommand{\E}{\mathrm{E}}
\newcommand{\Var}{\mathrm{Var}}
\newcommand{\Cov}{\mathrm{Cov}}
\newcommand{\Bias}{\mathrm{Bias}}

\begin{document}

\maketitle

\pagebreak

\begin{homeworkProblem}[1]

    Suppose that X1, X2, X3 are independent with the common probability mass function
    \[ P\{X_i = 0\} = .2,\quad 
        P\{X_i = 1\} = .3,\quad 
        P\{X_i = 3\} = .5, \quad
        i = 1, 2, 3 \]

    \begin{enumerate}[label=(\alph*)]
        \item Plot the probability mass function of 
            $ \overline{X_2} = \frac{X_1 + X_2}{2} $.

            
        \item Determine $\E [\overline{X_2}]$ and $\Var(\overline{X_2})$.  

            \[ \E [\overline{X_2}] = 1.8 \]
            \[ \Var(\overline{X_2}) = 0.8... \]

        \stepcounter{enumi}
        \item Determine $\E [\overline{X_3}]$ and $\Var(\overline{X_3})$.  

            \[ \E [\overline{X_3}] = 1.8 \]
            \[ \Var(\overline{X_3}) = 0.5... \]
    \end{enumerate}
    
\end{homeworkProblem}

\begin{homeworkProblem}[2]

    If 10 fair dice are rolled, approximate the probability that the sum of the values
    obtained (which ranges from 10 to 60) is between 30 and 40 inclusive.\\

    \textbf{Solution} 

    \begin{align*}
        P\{30\leq X\leq40\} &= P\{29.5\leq X \leq 40.5\} \\
                            &\cong P\left\{\frac{29.5-35}{\sqrt{\frac{350}{12}}}\leq Z \leq \frac{40.5-35}{\sqrt{\frac{350}{12}}}\right\} \\
                            &= \Phi(1.02) - \Phi(-1.02) = .6922
    \end{align*}
    
\end{homeworkProblem}

\begin{homeworkProblem}[3]

    Approximate the probability that the sum of 16 independent uniform (0, 1) random
    variables exceeds 10. \\

    \textbf{Solution} 

    \[ P\{S>10\} \cong 1-\Phi\left(\frac{2}{\sqrt{\frac{4}{3}}}\right) = .42 \]


\end{homeworkProblem}

\pagebreak

\begin{homeworkProblem}[5]

    A highway department has enough salt to handle a total of 80 inches of snowfall.
    Suppose the daily amount of snow has a mean of 1.5 inches and a standard
    deviation of .3 inches.

    \begin{enumerate}[label=(\alph*)]
        \item Approximate the probability that the salt on hand will suffice for the next 50 days.

            \[ \Phi\left(\frac{80-50(1.5)}{.3\sqrt{50}}\right) = \Phi(2.36...) = .99...\]
        \item What assumption did you make in solving part (a)?
            
            That the snowfall is independent of each other every day.

        \item Do you think this assumption is justified? Explain briefly

            Probably not, because snowstorms can last for multiple days etc. Weather is pretty cyclic.
    \end{enumerate}
    
\end{homeworkProblem}

\begin{homeworkProblem}[9]

    The lifetime of a certain electrical part is a random variable with mean 100 hours
    and standard deviation 20 hours. If 16 such parts are tested, find the probability
    that the sample mean is

    \begin{enumerate}[label=(\alph*)]
        \item less than 104;

            \[ P\{\overline{X} < 104\} \cong \Phi\left( \frac{52-60}{1.5\sqrt{52}} \right) = .23 \]
        \item between 98 and 104 hours.

            \[ .79 - \Phi(\frac{-8}{40}) = .443 \]
    \end{enumerate}
    
\end{homeworkProblem}

\begin{homeworkProblem}[10]

    A tobacco company claims that the amount of nicotine in its cigarettes is a random
    variable with mean 2.2 mg and standard deviation .3 mg. However, the
    sample mean nicotine content of 100 randomly chosen cigarettes was 3.1 mg.
    What is the approximate probability that the sample mean would have been as
    high or higher than 3.1 if the company’s claims were true?\\

    \textbf{Solution} 

    \[ 1-\Phi(\frac{9}{.3}) = 0 \]
    
\end{homeworkProblem}

\pagebreak

\begin{homeworkProblem}[12]

    An instructor knows from past experience that student exam scores have mean
    77 and standard deviation 15. At present the instructor is teaching two separate
    classes — one of size 25 and the other of size 64.

    \begin{enumerate}[label=(\alph*)]
        \item Approximate the probability that the average test score in the class of size 25 lies between 72 and 82.

            \[ \Phi(\frac{25}{15}) - \Phi(\frac{-25}{15}) = .90 \]
        \item Repeat part (a) for a class of size 64.

            \[ \Phi(\frac{40}{15}) - \Phi(\frac{-40}{15}) = .99 \]
        \item What is the approximate probability that the average test score in the class of size 25 is higher than that of the class of size 64?

            \[.5\]
        \item Suppose the average scores in the two classes are 76 and 83. Which class, the one of size 25 or the one of size 64, do you think was more likely to have averaged 83?

            The class of size 25.
    \end{enumerate}
    
\end{homeworkProblem}

\begin{homeworkProblem}[14]

    Each computer chip made in a certain plant will, independently, be defective
    with probability .25. If a sample of 1,000 chips is tested, what is the approximate
    probability that fewer than 200 chips will be defective?\\

    \textbf{Solution} 
    
    \[ P\{X < 199.5\} \cong \Phi\left(\frac{-50.5}{\sqrt{\frac{3000}{16}}}\right) = \Phi(-3.7...) = .0001 \]
\end{homeworkProblem}

\pagebreak

\begin{homeworkProblem}[15]

    A club basketball team will play a 60-game season. Thirty-two of these games
    are against class A teams and 28 are against class B teams. The outcomes of
    all the games are independent. The team will win each game against a class
    A opponent with probability .5, and it will win each game against a class B
    opponent with probability .7. Let X denote its total number of victories in the
    season.

    \begin{enumerate}[label=(\alph*)]
        \item Is X a binomial random variable?

            It is not a binomial random variable.

        \item Let $X_A$ and $X_B$ denote, respectively, the number of victories against class A and class B teams. What are the distributions of $X_A$ and $X_B$?

            Both are binomially distributed.

        \item What is the relationship between $X_A$, $X_B$, and $X$?

            X is the sum of the other two random variables.

        \item Approximate the probability that the team wins 40 or more games.

            \[ P\{X > 39.5\} \cong .148 \]
    \end{enumerate}
    
\end{homeworkProblem}

\begin{homeworkProblem}[17]

    Use the text disk to compute $P\{X \leq 10\}$ when X is a binomial random variable
    with parameters n = 100, p = .1. Now compare this with its 
    
    \begin{enumerate}[label=(\alph*)]
        \item Poisson approximation

            \[ .583 \]
        \item Normal approximation

            \[ .566 \]
    \end{enumerate}
        In using the normal approximation, write the desired probability as $P\{X < 10.5\}$ so as to utilize the continuity correction.
    
\end{homeworkProblem}

\begin{homeworkProblem}[18]

    The temperature at which a thermostat goes off is normally distributed with variance
    $\sigma^2$. If the thermostat is to be tested five times, find

    \begin{enumerate}[label=(\alph*)]
        \item $P\{\frac{S^2}{\sigma^2} \leq 1.8\}$

            \[ P\{\chi_4^2 \leq 7.2\} \]

        \item $P\{.85 \leq \frac{S^2}{\sigma^2} \leq 1.15\}$

            \[ P\{3.4 \leq \chi_4^2 \leq 4.2\} \]
    \end{enumerate}

    where $S^2$ is the sample variance of the five data values.
    
\end{homeworkProblem}

\pagebreak

\begin{homeworkProblem}[20]

    Consider two independent samples — the first of size 10 from a normal population
    having variance 4 and the second of size 5 from a normal population having
    variance 2. Compute the probability that the sample variance from the second
    sample exceeds the one from the first. (Hint: Relate it to the F-distribution.)

    \textbf{Solution} 

    \[ P\{S_2^2 > S_1^2\} = P\{F_{9,4} < 0.5 \}\]

\end{homeworkProblem}

\begin{homeworkProblem}[21]

    Twelve percent of the population is left-handed. Find the probability that there
    are between 10 and 14 left-handers in a random sample of 100 members of this
    population. That is, find $P\{10 \leq X \leq 14\}$, where X is the number of left-handers in the sample.

    \textbf{Solution} 
    
    \[ .56... \]

\end{homeworkProblem}

\begin{homeworkProblem}[23]

    The following table gives the percentages of individuals of a given city, categorized by gender, that follow certain negative health practices. \\
    
    \begin{tabular}{ l c c c c } \hline
        & Sleeps 6 Hours or Less per Night & Smoker & Rarely Eats Breakfast & Is 20 Percent or More Overweight \\ \hline
        Men     &   22.7    &   28.4    &   45.4    &   29.6 \\
        Women   &   21.4    &   22.8    &   42.0    &   25.6 \\ \hline
    \end{tabular}\\

    Suppose a random sample of 300 men is chosen. Approximate the probability that

    \begin{enumerate}[label=(\alph*)]
        \item at least 150 of them rarely eat breakfast

            \[.6711\]

        \item fewer than 100 of them smoke

            \[.9735\]
    \end{enumerate}
    
\end{homeworkProblem}

\begin{homeworkProblem}[25]

    (Use the table from Problem 23.) Suppose random samples of 300 women and
    of 300 men are chosen. Approximate the probability that more women than men
    rarely eat breakfast.\\

    \textbf{Solution} 

    \[ P\{X-Y>0\} \cong \Phi(-.84) = .2005 \]
    
\end{homeworkProblem}

\pagebreak

\begin{homeworkProblem}[28]

    The sample mean and sample standard deviation of all San Francisco student
    scores on the most recent Scholastic Aptitude Test examination in mathematics
    were 517 and 120. Approximate the probability that a random sample of 144
    students would have an average score exceeding

    \begin{enumerate}[label=(\alph*)]
        \item 507

            \[ .84 \] 

        \item 517

            \[ .02 \] 

        \item 537

            \[ .00 \] 

        \item 550

            \[ .06 \] 

    \end{enumerate}
    
\end{homeworkProblem}

\begin{homeworkProblem}[29]

    The average salary of newly graduated students with bachelor’s degrees in chemical
    engineering is \$53,600, with a standard deviation of \$3,200. Approximate the
    probability that the average salary of a sample of 12 recently graduated chemical
    engineers exceeds \$55,000.\\

    \textbf{Solution} 

    \[ .06 \]

\end{homeworkProblem}

\end{document}
