\documentclass{article}

\usepackage{fancyhdr}
\usepackage{extramarks}
\usepackage{amsmath}
\usepackage{amsthm}
\usepackage{amsfonts}
\usepackage[plain]{algorithm}
\usepackage{enumerate}
\usepackage{mathtools}
\usepackage{enumitem}
\usepackage{lscape}

%
% Basic Document Settings
%

\topmargin=-0.45in
\evensidemargin=0in
\oddsidemargin=0in
\textwidth=6.5in
\textheight=9.0in
\headsep=0.25in

\linespread{1.1}

\pagestyle{fancy}
\lhead{\hmwkAuthorName}
\chead{\hmwkClass\ (\hmwkClassInstructor\ \hmwkClassTime): \hmwkTitle}
\rhead{\firstxmark}
\lfoot{\lastxmark}
\cfoot{\thepage}

\renewcommand\headrulewidth{0.4pt}
\renewcommand\footrulewidth{0.4pt}

\setlength\parindent{0pt}

%
% Homework Details
%   - Title
%   - Due date
%   - Class
%   - Section/Time
%   - Instructor
%   - Author
%

\newcommand{\hmwkTitle}{Homework \#2}
\newcommand{\hmwkDueDate}{February 10, 2017}
\newcommand{\hmwkClass}{CS 320}
\newcommand{\hmwkClassInstructor}{Professor Dmitry Ponomarev}
\newcommand{\hmwkAuthorName}{Tim Hung}

%
% Title Page
%

\title{
    \vspace{2in}
    \textmd{\textbf{\hmwkClass:\ \hmwkTitle}}\\
    \normalsize\vspace{0.1in}\small{Due\ on\ \hmwkDueDate\ at 10:50pm}\\
    \vspace{0.1in}\large{\textit{\hmwkClassInstructor\ \hmwkClassTime}}\\
}

\author{\textbf{\hmwkAuthorName}}
\date{}

\renewcommand{\part}[1]{\textbf{\large Part \Alph{partCounter}}\stepcounter{partCounter}\\}

%
% Various Helper Commands
%

% Useful for algorithms
\newcommand{\alg}[1]{\textsc{\bfseries \footnotesize #1}}

% For derivatives
\newcommand{\deriv}[1]{\frac{\mathrm{d}}{\mathrm{d}x} (#1)}

% For partial derivatives
\newcommand{\pderiv}[2]{\frac{\partial}{\partial #1} (#2)}

% Integral dx
\newcommand{\dx}{\mathrm{d}x}

% Alias for the Solution section header
\newcommand{\solution}{\textbf{\large Solution}}

% Probability commands: Expectation, Variance, Covariance, Bias
\newcommand{\E}{\mathrm{E}}
\newcommand{\Var}{\mathrm{Var}}
\newcommand{\Cov}{\mathrm{Cov}}
\newcommand{\Bias}{\mathrm{Bias}}

\begin{document}

\maketitle

\pagebreak


{\large \textbf{Problem Statement: } Apply a 2-level carry-lookahead addition algorithm discussed in class to add the following two 16-bit numbers:}


\begin{center}
    \begin{tabular}{ r l c r l }
        & $0011010000110111_2$ & & & $13367_{10}$\\
        + & $0011110011011110_2$ & & + & $15582_{10}$\\
        \hline
    \end{tabular}
\end{center}

\textbf{Bit level propogation and generation}
\[
    g_i = a_i \cdot b_i
\]
\[
    p_i = a_i \oplus b_i
\]

\textbf{Group level propogation and generation}
\[
    G_i = \prod_{x = 4i}^{4i+3}g_x
\]
\[
    P_i = 
    \begin{cases}
    1 & \text{if } $g_{4i+3} = 1$\\        
        1 & \text{if earlier generate is true and all intermediate propagates are true}\\
        0 & \text{otherwise}
    \end{cases}
\]
Values are displayed in the table below.\\

\textbf{Calculating group level carries:} $c_{i+1} = g_i + p_i \cdot c_i$
\begin{equation} \label{eq1}
    \begin{align}
        C_0 & = G_0 + P_0 \cdot C_0 & = 1 \\
        C_1 & = G_1 + P_1 \cdot C_1 & = 1 \\
        C_2 & = G_2 + P_2 \cdot C_2 & = 1 \\
        C_3 & = G_3 + P_3 \cdot C_3 & = 0 \\
    \end{align}
\end{equation}

\textbf{Calculating bit level carries:} $c_{i+1} = g_i + p_i \cdot c_i$
\begin{align*}
    c_1    &= g_0    + p_0    \cdot c_0    &= 0\\
    c_2    &= g_1    + p_1    \cdot c_1    &= 1\\
    c_3    &= g_2    + p_2    \cdot c_2    &= 1\\
    c_4    &= g_3    + p_3    \cdot C_0    &= 1\\
    c_5    &= g_4    + p_4    \cdot c_4    &= 1\\
    c_6    &= g_5    + p_5    \cdot c_5    &= 1\\
    c_7    &= g_6    + p_6    \cdot c_6    &= 1\\
    c_8    &= g_7    + p_7    \cdot C_1    &= 1\\
    c_9    &= g_8    + p_8    \cdot c_8    &= 0\\
    c_{10} &= g_9    + p_9    \cdot c_9    &= 0\\
    c_{11} &= g_{10} + p_{10} \cdot c_{10} &= 1\\
    c_{12} &= g_{11} + p_{11} \cdot C_2    &= 1\\
    c_{13} &= g_{12} + p_{12} \cdot c_{12} &= 1\\
    c_{14} &= g_{13} + p_{13} \cdot c_{13} &= 1\\
    c_{15} &= g_{14} + p_{14} \cdot c_{14} &= 0\\
\end{align}

\pagebreak

\textbf{Remaining calculations are displayed in the table}\\
{\small Least significant bit is on the left}
\begin{center}
    \begin{tabular}{ | c | c | c | c | c | c | c | c | c | c | c | c | c | c | c | c | c | c | c | c | c |}
        \hline
        bit & 0 & 1 & 2 & 3 &  & 4 & 5 & 6 & 7 &   & 8 & 9 & 10& 11&   & 12& 13& 14& 15& 16\\
        \hline\hline
        a_i & 1 & 1 & 1 & 0 &  & 1 & 1 & 0 & 0 &   & 0 & 0 & 1 & 0 &   & 1 & 1 & 0 & 0 & \\
        b_i & 0 & 1 & 1 & 1 &  & 1 & 0 & 1 & 1 &   & 0 & 0 & 1 & 1 &   & 1 & 1 & 0 & 0 & \\
        \hline\hline
        g_i & 0 & 1 & 1 & 0 &  & 1 & 0 & 0 & 0 &   & 0 & 0 & 1 & 0 &   & 1 & 1 & 0 & 0 & \\
        p_i & 1 & 0 & 0 & 1 &  & 0 & 1 & 1 & 1 &   & 0 & 0 & 0 & 1 &   & 0 & 0 & 0 & 0 & \\
        \hline\hline
        G_i &   &   &   &   & 1&   &   &   &   & 1 &   &   &   &   & 1 &   &   &   &   & 0\\
        P_i &   &   &   &   & 0&   &   &   &   & 0 &   &   &   &   & 0 &   &   &   &   & 0\\
        \hline\hline
        c_i &   & 0 & 1 & 1 &  & 1 & 1 & 1 & 1 &   & 1 & 0 & 0 & 1 &   & 1 & 1 & 1 & 0 & \\
        C_i &   &   &   &   & 1&   &   &   &   & 1 &   &   &   &   & 1 &   &   &   &   & 0\\
        \hline\hline
        S_i &1  & 0 & 1 & 0 &  & 1 & 0 & 0 & 0 &   & 1 & 0 & 0 & 0 &   & 1 & 1 & 1 & 0 & \\
        \hline
    \end{tabular}
\end{center}

\vspace{1in}

\textbf{Ripple carry adder}
{\small Least significant bit is on the right}
\begin{center}
    \begin{tabular}{ | c | c | c | c | c | c | c | c | c | c | c | c | c | c | c | c | c | c |}
        \hline
        bit & 15 & 14 & 13 & 12 & 11 & 10 & 9 & 8 & 7 & 6 & 5 & 4 & 3 & 2 & 1 & 0 & \\
        \hline \hline
        carry &   & 1 & 1 & 1 & 1 &   &   & 1 & 1 & 1 & 1 & 1 & 1 & 1 &   &   & \\
        \hline
        a_i   & 0 & 0 & 1 & 1 & 0 & 1 & 0 & 0 & 0 & 0 & 1 & 1 & 0 & 1 & 1 & 1 & \\
        b_i   & 0 & 0 & 1 & 1 & 1 & 1 & 0 & 0 & 1 & 1 & 0 & 1 & 1 & 1 & 1 & 0 & \\
        \hline
        sum   & 0 & 1 & 1 & 1 & 0 & 0 & 0 & 1 & 0 & 0 & 0 & 1 & 0 & 1 & 0 & 1 & \\
        \hline
    \end{tabular}
\end{center}\\

\vspace{1in}

\begin{center}
    \begin{tabular}{ r l c r l }
        & $0011010000110111_2$ & & & $13367_{10}$\\
        + & $0011110011011110_2$ & & + & $15582_{10}$\\
        \hline
        = & $0111000100010101_2$ & & = & $28949_{10}$\\
    \end{tabular}
\end{center}

\textbf{Conclusion:}\\
Ripple Carry Adder and Carry-Lookahead Adder both generate the same accurate result!

\pagebreak


\end{document}
