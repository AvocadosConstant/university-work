\documentclass{article}

\usepackage{fancyhdr}
\usepackage{extramarks}
\usepackage{amsmath}
\usepackage{amsthm}
\usepackage{amsfonts}
\usepackage{tikz}
\usepackage{enumerate}
\usepackage{mathtools}
\usepackage{enumitem}
\usepackage{pgfplots}
\usepackage{diagbox}

%
% Basic Document Settings
%

\topmargin=-0.45in
\evensidemargin=0in
\oddsidemargin=0in
\textwidth=6.5in
\textheight=9.0in
\headsep=0.25in

\linespread{1.1}

\pagestyle{fancy}
\lhead{\hmwkAuthorName}
\chead{\hmwkClass\ (\hmwkClassInstructor\ \hmwkClassTime): \hmwkTitle}
\rhead{\firstxmark}
\lfoot{\lastxmark}
\cfoot{\thepage}

\renewcommand\headrulewidth{0.4pt}
\renewcommand\footrulewidth{0.4pt}

\setlength\parindent{0pt}

%
% Create Problem Sections
%

\newcommand{\enterProblemHeader}[1]{
    \nobreak\extramarks{}{Problem \arabic{#1} continued on next page\ldots}\nobreak{}
    \nobreak\extramarks{Problem \arabic{#1} (continued)}{Problem \arabic{#1} continued on next page\ldots}\nobreak{}
}

\newcommand{\exitProblemHeader}[1]{
    \nobreak\extramarks{Problem \arabic{#1} (continued)}{Problem \arabic{#1} continued on next page\ldots}\nobreak{}
    \stepcounter{#1}
    \nobreak\extramarks{Problem \arabic{#1}}{}\nobreak{}
}

\setcounter{secnumdepth}{0}
\newcounter{partCounter}
\newcounter{homeworkProblemCounter}
\setcounter{homeworkProblemCounter}{1}
\nobreak\extramarks{Problem \arabic{homeworkProblemCounter}}{}\nobreak{}

%
% Homework Problem Environment
%
% This environment takes an optional argument. When given, it will adjust the
% problem counter. This is useful for when the problems given for your
% assignment aren't sequential. See the last 3 problems of this template for an
% example.
%
\newenvironment{homeworkProblem}[1][-1]{
    \ifnum#1>0
    \setcounter{homeworkProblemCounter}{#1}
    \fi
    \section{Problem \arabic{homeworkProblemCounter}}
    \setcounter{partCounter}{1}
    \enterProblemHeader{homeworkProblemCounter}
    }{
    \exitProblemHeader{homeworkProblemCounter}
}

%
% Homework Details
%   - Title
%   - Due date
%   - Class
%   - Section/Time
%   - Instructor
%   - Author
%

% TODO: Replace with correct number and date
\newcommand{\hmwkTitle}{Homework\ \#0}
\newcommand{\hmwkDueDate}{October 13, 2017}
\newcommand{\hmwkClass}{CS 436}
\newcommand{\hmwkClassInstructor}{Professor Arti Ramesh}
\newcommand{\hmwkAuthorName}{Tim Hung}
\newcommand{\hmwkBNumber}{B00518486}

%
% Title Page
%

\title{
    \vspace{2in}
    \textmd{\textbf{\hmwkClass:\ \hmwkTitle}}\\
    \vspace{0.1in}\large{\textit{\hmwkClassInstructor\ \hmwkClassTime}}\\
    \vspace{0.5in}\textbf{Academic Honesty Pledge}\\
    I have done this assignment completely on my own. I have not copied it, nor have I given my solution to anyone else. I understand that if I am involved in plagiarism or cheating I will have to sign an official form that I have cheated and that this form will be stored in my official university record. I also understand that I will receive a grade of 0 for the involved assignment for my first offense and that I will receive a grade of “F” for the course for any additional offense.\\
    \vspace{0.4in}\\
}

\author{
    \textbf{\hmwkAuthorName}\\
    \hmwkBNumber
}
\date{}

\renewcommand{\part}[1]{\textbf{\large Part \Alph{partCounter}}\stepcounter{partCounter}\\}

%
% Various Helper Commands
%

% Useful for algorithms
\newcommand{\alg}[1]{\textsc{\bfseries \footnotesize #1}}

% For derivatives
\newcommand{\deriv}[1]{\frac{\mathrm{d}}{\mathrm{d}x} (#1)}

% For partial derivatives
\newcommand{\pderiv}[2]{\frac{\partial}{\partial #1} (#2)}

% Integral dx
\newcommand{\dx}{\mathrm{d}x}

% Alias for the Solution section header
\newcommand{\solution}{\textbf{\large Solution}}

% Probability commands: Expectation, Variance, Covariance, Bias
\newcommand{\E}{\mathrm{E}}
\newcommand{\Var}{\mathrm{Var}}
\newcommand{\Cov}{\mathrm{Cov}}
\newcommand{\Bias}{\mathrm{Bias}}

\begin{document}

\maketitle

\pagebreak


\begin{homeworkProblem}[]

    Derive maximum likelihood estimators for parameter p based on a Bernoulli(p) sample of size n. \\

    \textbf{Solution} 
    \[ \hat{p} = \bar{X}\]

\end{homeworkProblem}

\begin{homeworkProblem}[]

    Derive maximum likelihood estimators for parameter p based on a Binomial(N, p) sample of size n. Compute your estimators if the observed sample is (3, 6, 2, 0, 0, 3) and N = 10.\\

    \textbf{Solution} 
    \[ \hat{p} = \frac{\bar{X}}{N}\]

    \[ \hat{p} = \frac{\frac{3+6+2+0+0+3}{6}}{10} = \frac{\frac{14}{6}}{10} = \frac{140}{6} = \frac{140}{6} = 23.33...\]


\end{homeworkProblem}

\begin{homeworkProblem}[]

    Derive maximum likelihood estimators for parameters a and b based on a Uniform (a, b) sample of size n. \\

    \textbf{Solution} 
    \[ \hat{a} = \text{smallest } X_1\]
    \[ \hat{b} = \text{largest } X_1\]

\end{homeworkProblem}

\begin{homeworkProblem}[]

    Derive maximum likelihood estimators for parameter $\mu$ based on a $Normal(\mu, \sigma^2)$ sample of size n with known variance $\sigma^2$ and unknown mean $\mu$.\\

    \textbf{Solution} 
    \[ \hat{\mu} = \frac{\sum{X_i}}{N}\]

\end{homeworkProblem}

\begin{homeworkProblem}[]

    Derive maximum likelihood estimators for parameter $\sigma$ based on a $Normal(\mu, \sigma^2)$ sample of size n with known mean $\mu$ and unknown variance $\sigma^2$.\\

    \textbf{Solution} 
    \[ \hat{\sigma} = \sqrt{\frac{\sum{X_i - \mu}}{n}}\]

\end{homeworkProblem}

\begin{homeworkProblem}[]

    Derive maximum likelihood estimators for parameters $(\mu, \sigma^2)$ based on a $Normal(\mu, \sigma^2)$ sample of size n with unknown mean $\mu$ and variance $\sigma^2$.\\

    \textbf{Solution} 
    \[ \hat{\mu} = \text{Sample mean}\]
    \[ \hat{\sigma} = \text{Sample standard deviation}\]

\end{homeworkProblem}

\end{document}
