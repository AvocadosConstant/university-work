\documentclass{beamer}
\usepackage{apacite}
\usepackage{filecontents}
\usepackage{graphicx}

\title{CS 373 Presentation}
\subtitle{A proof of a context free language construction}
\author{Tim Hung}
\institute{Binghamton University}
\date{April 6, 2016}

\usetheme{Copenhagen}
\setbeamertemplate{itemize items}[default]
\setbeamertemplate{enumerate items}[default]

\setbeamertemplate{section page}
{
  \begin{centering}
    \begin{beamercolorbox}[sep=12pt,center]{part title}
      \usebeamerfont{section title}\insertsection\par
    \end{beamercolorbox}
  \end{centering}
}
\setbeamertemplate{caption}{\raggedright\insertcaption\par}

\begin{document}
\frame{\titlepage}
\section{Overview}\frame{\sectionpage}
\begin{frame}{Overview}
  Let A be a regular language and B a context free language. \\
  Is $A\cap∩ B$ a context free language?
\end{frame}

\section{Solution}\frame{\sectionpage}
\subsection{Overview}
\begin{frame}{Solution}
    \begin{itemize}
        \item \textbf{Yes, $A\cap∩ B$ is a context free language}
        \item How will we prove it?
        \begin{itemize}
            \item We will construct a PDA that simulates the action of a FA and a PDA
        \end{itemize}
    \end{itemize}
\end{frame}

\subsection{Proof}
\begin{frame}{Construction}
    \begin{itemize}
    \item Let $M_A$ be a DFA such that $L(M_A) = A$
        \begin{itemize}
            \item $M_A = (Q_A, \Sigma, \delta_A, q_{A0}, F_A)$
        \end{itemize}
    \item Let $M_B$ be a PDA such that $L(M_B) = B$\\
        \begin{itemize}
            \item $M_B = (Q_B, \Sigma, \Gamma, \delta_B, q_{B0}, F_B)$.
        \end{itemize}

    \item Define PDA $M = (Q_A\times Q_B, \Sigma, \Gamma, \delta, (q_{A0}, q_{B0}), F_A\times F_B)$
    \end{itemize}
    
    \[
        \delta(q_A, q_B, a, b)=\left\{
        \begin{array}{ll}
            (\delta_A(q_A, a), \delta_B(q_B, a, b)) & a\ne \varepsilon\\
            (q_A, \delta_B(q_B, a, b))  & a=\varepsilon\\
        \end{array}
        \right.
    \]
    where $q_A$ and $q_B$ are states from $Q_A$ and $Q_B$, a is from $\Sigma_\varepsilon$ and b is from $\Gamma_\varepsilon$.
\end{frame}

\begin{frame}{Proof}
    \begin{itemize}
        \item Let $w\in A\cap B$. \\

        \item Then there are sequences of states from $Q_A$ and $Q_B$ such that the sequences start in the start states, 
        end in an accept state, and do valid transitions.

        \item Thus both $M_A$ and $M_B$ end in accept states, and likewise the sequence in $M$ starts in the start state,  
        end in an accept state, and do valid transitions. 

        \item Thus w is accepted by $M$ and $A\cap B\subset L(M)$.
    \end{itemize}
\end{frame}

\begin{frame}{Proof cont.}
    \begin{itemize}
        \item Let $w\in L(M)$.

        \item Then there is a sequence of states that $M$ transitions through while processing $w$, 
        starting in the start state, ending in an accept state, and performing valid transitions.

        \item By the definition of $M$, there are sequences from $M_A$ and $M_B$ that do the same. 
        Thus $w$ is accepted by $M_A$ and $M_B$. 

        \item The $w\in A$ and $w\in B$. Thus $L(M) \subset A\cap B$.

        \item Thus we have a PDA, $M$, that accepts $A\cap B$ 
        
        \item Therefore $A\cap B$ is context free.
    \end{itemize}
\end{frame}

\section{Summary}\frame{\sectionpage}
\begin{frame}{Recap}
    \begin{itemize}
        \item Given a regular language $A$ and a context free language $B$ \\
        \begin{itemize}
            \item We want to show $A\cap∩ B$ is a context free language
        \end{itemize}
        \item Construction 
            \begin{itemize}
                \item We constructed $M_A$ a DFA that recognizes regular language A
                \item We constructed $M_B$ a PDA that recognizes regular language B.
            \end{itemize}
        \item Proof
            \begin{itemize}
                \item We show that processing a $w\in A\cap B$ results in valid transitions leading to accept states in both $M_A$ and $M_B$.
                    \begin{itemize}
                        \item Therefore $A\cap B\subset L(M)$
                    \end{itemize}
                \item We show that processing a $w\in L(M)$ results in valid transitions leading to accept states in $M$.
                    \begin{itemize}
                        \item Therefore $L(M) \subset A\cap B$
                    \end{itemize}
            \end{itemize}
        \item Therefore, $M$ accepts $A\cap B$ so $A\cap B$ is context free.
    \end{itemize}
\end{frame}

\begin{frame}{Questions}
    Thank you.\\
    Any questions?
\end{frame}

\end{document}
\grid
