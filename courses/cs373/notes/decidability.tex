\sektion{1}{Decidability}

Hello welcome to the section.

\begin{corollary}
    A language is decidable if $\exists$ a non deterministic Turing Machine that recognizes it.
\end{corollary}

\begin{theorem}
    A language is Turing Recognizable if and only if some enumerator enumerates it.
\end{theorem}

\begin{theorem}
    The class of Fontext Free Languages is a proper subset of the Turing Recognizable languages.
\end{theorem}

Hilbert's 10th Problem: Given a polynomial with integer coeficients, does there exist an integer root to that polynomial.

\[D = \{p|\text{p is a polynomial over one variable}\}  \]
\[F = \{p|\text{p is a polynomial over one or more variables}\}  \]

\begin{theorem}
The class of Turing Recognizable Languages is closed under $\cup$.
\end{theorem}

\begin{proof}
    Let A, B be Turing Recognizable Languages.

    $\exists$ Turing Machines $M_A, M_B, L(M_A)=A, L(M_B)=B$.

    We want Turing Machine M such that $L(M)=A\cup B$

    On input w, M does:

    1. run $M_A$ and $M_B$ in parallel on w

        -if $M_A$ or $M_B$ then halt and accept
        
        -if $M_A$ and $M_B$ then halt and reject

    \begin{claim}
        $L(M)=A\cup B$
    \end{claim}
    Let $w\in L(M)$
    $w\in A\cup B$ etc.....
\end{proof}





\begin{definition}
    An algorithm is a well defined sequence of steps to perform a computation.
\end{definition}


\begin{definition}
    A Universal Turing Machine ($\mathit{u}$) is a Turing Machine that can simulate running any Turing Machine on an input string.
\end{definition}

\pagebreak

%---------------------------------- DFA
\subsektion{Acceptance problem for DFAs}
    \begin{definition}
        The acceptance problem for Deterministic Finite Automata is $A_{DFA}=\{<M,w> | M$ is a $DFA$ and $w\in L(M)\}$.
    \end{definition}

    \begin{theorem}
        $A_{DFA}$ is decidable.
    \end{theorem}

    \begin{proof}
        On an input $<M,w>$, $X$ does:

        1. Simulate running M for $|w|$ transitions.

        2. If $M$ is in an accept state, halt and accept.
        
        3. Halt and reject.

        $<M,w>\in A_{DFA} \Rightarrow L(M) \Rightarrow X$ Halt and accept $\Rightarrow <M,w>\in L(X)$

        $<M,w>\in L(X) \Rightarrow w\in L(M) \Rightarrow <M,w>\in A_{DFA}$

        $M=(Q,\Sigma, \delta, q_0, F)$

            $|Q|<\infty$

            $|\Sigma|<\infty$

            $|\delta|<\infty$

        Therefore X decides $A_{DFA}$.
    \end{proof}

%---------------------------------- NFA
\subsektion{Acceptance problem for NFAs}
    \begin{definition}
        The acceptance problem for Non-deterministic Finite Automata is $A_{NFA}=\{<M,w> | M$ is a $NFA$ and $w\in L(M)\}$.
    \end{definition}

    \begin{theorem}
        $A_{NFA}$ is decidable.
    \end{theorem}

    \begin{proof}
        On an input $<M,w>$, $Y$ does:

        1. Construct DFA $M'$ such that $L(M')=L(M)$.

        2. Call Turing Machine X with input $<M,w>$ and return what it returns.


        Let $<M,w>\in A_{NFA} 
            \Rightarrow w\in L(M) 
            \Rightarrow L(M') 
            \Rightarrow <M',w>\in A_{DFA} 
            \Rightarrow Y$ accepts $<M,w> 
            \Rightarrow <M,w>\in L(Y)$

        $<M,w>\in L(Y) 
            \Rightarrow <M',w>\in L(X) = A_{DFA} 
            \Rightarrow w\in L(M') = L(M) 
            \Rightarrow <M,w>\in A_{NFA}$

        $M=(Q,\Sigma, \delta, q_0, F)$

            $|Q|<\infty$

            $\mathcal{P}(Q)<\infty$

            $|\Sigma|<\infty$

            $|\delta|<\infty$

        Therefore Y decides $A_{NFA}$.
    \end{proof}

%---------------------------------- Regexp
\subsektion{Generation problem for Regular Expressions}
    \begin{definition}
        The generation problem for regular expressions $A_{rex}=\{<R,w>|R$ is a regular expression and $w\in L(R)\}$.
    \end{definition}

    \begin{theorem}
        $A_{rex}$ is decidable.
    \end{theorem}

    \begin{proof}
        On an input $<M,w>$, $Z$ does:

        1. Construct NFA $N$ such that $L(N)=L(R)$.

        2. Call Turing Machine Y with input $<N,w>$ and return what it returns.


        Want to prove that $L(Z)=A_{rex}$.

        Therefore Y decides $A_{NFA}$.
    \end{proof}

\subsektion{Emptiness problem for DFAs}
    \begin{definition}
        The emptiness problem for DFAs is $E_{DFA} = \{<M>| M$ is a DFA and $L(M)=\emptyset \}$.
    \end{definition}

    \begin{theorem}
        $E_{DFA}$ is decidable.
    \end{theorem}



%   April 19, 2016
\begin{theorem}
    There are languages that are not Turing recognizable.
\end{theorem}

    -   There are a countably infinite number of Turing machines

    -   We want to show there are an uncountable number of languages over $\{0,1\}$


%   April 21, 2016

\begin{theorem}
    Every context free language is decidable.
\end{theorem}


\subsektion{Equality problem for CFGs}
    \begin{definition}
        The equality problem for context free grammars is 
    \end{definition}

    \begin{theorem}
        $EQ_{CFG}$ is not decidable.
    \end{theorem}


\subsektion{Co-Turing Recognizablilty}

\begin{definition}
    A language is called Co-Turing recognizable if some Turing machine recognizes its complement.
\end{definition}

\begin{theorem}
    A language is decidable if and only if it is both Turing recognizable and Co-Turing recognizable.
\end{theorem}

\begin{corollary}
    $A_{TM}$ is not co-Turing Recognizable.
\end{corollary}
