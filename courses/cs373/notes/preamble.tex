\documentclass[12pt,twoside]{article}
%%%%%%% Pagestyle stuff %%%%%%%%%%%%%%%%%
 \usepackage{fancyhdr}                 %%
   \pagestyle{fancy}                   %%
   \fancyhf{} %delete the current section for header and footer
 \usepackage[paperheight=11in,%        %%
             paperwidth=8in,%        %%
             outer=1in,%             %%
             inner=1in,%             %%
             bottom=.7in,%             %%
             top=.7in,%                %%
             includeheadfoot]{geometry}%%
%    \columnsep 3em
   \setlength{\headwidth}{6in}       %% 
   \fancyhead[RO,LE]{\thepage}         %%
   \fancyhead[LO,RE]{\sectionname}     %%
   \setlength{\headheight}{14.5pt}     %%
   \raggedbottom                       %%
%%%%%%% End Pagestyle stuff %%%%%%%%%%%%%
 
 \usepackage{etex}          % For some reason, pdflatex breaks if I don't include the etex package
 \usepackage{amsmath}       % I think this gives me some symbols
 \usepackage{amsthm}        % Does theorem stuff
 \usepackage{amssymb}       % more symbols and fonts
%  \usepackage{accents}
%  \usepackage{bbm}           % for the nice blackboard bold 1
 \usepackage{empheq}        % Some more extensible arrows, like \xmapsto
 \usepackage{enumitem}
 \usepackage{mathrsfs}      % Sheafy font \mathscr{}
% \usepackage{picinpar}      % for pictures in paragraphs
% \usepackage{tikz}
 \usepackage[nofancy]{rcsinfo}
 \usepackage[all]{xy}       % Include XY-pic
    \SelectTips{cm}{10}     % Use the nicer arrowheads
    \everyxy={<2.5em,0em>:} % Sets the scale I like
 \usepackage[colorlinks,
             linkcolor=black,
             pagebackref,
             bookmarksnumbered=true]{hyperref}

%%%%%%% Stuff for keeping track of sections %%%%%%%%%%%%%%%%%%%%%%%%%%%%
 \newcount\n
 \newcommand{\sektion}[2]{\stepcounter{section} \renewcommand{\thesection}{#1}\n=\count0 \newpage
\ifnum\count0=\n \ifnum\count0>1\ \newpage \fi\fi\section{#2} \gdef\sectionname{#1\quad #2, v.~\rcsInfoMonth-\rcsInfoDay}}
 \newcommand{\subsektion}[1]{\subsection*{#1} \addcontentsline{toc}{subsection}{#1}}
 %% This is the empty section title, before any section title is set %%
 \newcommand\sectionname{}                                           %%
%%%%%%% End stuff for keeping track of sections %%%%%%%%%%%%%%%%%%%%%%%

%%%%%%%%%%%%%%% Theorem Styles and Counters %%%%%%%%%%%%%%%%%%%%%%%%%%
 \renewcommand{\theequation}{\thesection.\arabic{equation}}         %%
 \makeatletter                                                      %%
    \@addtoreset{equation}{section} % Make the equation counter reset each section
    \@addtoreset{footnote}{section} % Make the footnote counter reset each section
                                                                    %%
 \newenvironment{warning}[1][]{%                                    %%
    \begin{trivlist} \item[] \noindent%                             %%
    \begingroup\hangindent=2pc\hangafter=-2                         %%
    \clubpenalty=10000%                                             %%
    \hbox to0pt{\hskip-\hangindent\manfntsymbol{127}\hfill}\ignorespaces%
    \refstepcounter{equation}\textbf{Warning~\theequation}%         %%
    \@ifnotempty{#1}{\the\thm@notefont \ (#1)}\textbf{.}            %%
    \let\p@@r=\par \def\p@r{\p@@r \hangindent=0pc} \let\par=\p@r}%  %%
    {\hspace*{\fill}$\lrcorner$\endgraf\endgroup\end{trivlist}}     %%
                                                                    %%
 \newenvironment{exercise}[1][]{\begin{trivlist}%                   %%
    \item{\bf Exercise\@ifnotempty{#1}{ #1}. }\it}{\end{trivlist}}   %%
 \newenvironment{solution}{\begin{trivlist}%                        %%
    \item{\it Solution.}}{\end{trivlist}}                           %%
                                                                    %%
 \def\newprooflikeenvironment#1#2#3#4{%                             %%
      \newenvironment{#1}[1][]{%                                    %%
          \refstepcounter{equation}                                 %%
          \begin{proof}[{\rm\csname#4\endcsname{#2~\theequation}%   %%
          \@ifnotempty{##1}{\the\thm@notefont \ (##1)}\csname#4\endcsname{.}}]%%
          \def\qedsymbol{#3}}%                                      %%
         {\end{proof}}}                                             %%
 \makeatother                                                       %%
                                                                    %%
 \newprooflikeenvironment{definition}{Definition}{$\diamond$}{textbf}%
 \newprooflikeenvironment{example}{Example}{$\diamond$}{textbf}     %%
 \newprooflikeenvironment{remark}{Remark}{$\diamond$}{textbf}       %%
                                                                    %%
 \theoremstyle{plain}                                               %%
 \newtheorem{theorem}[equation]{Theorem}                            %%
 \newtheorem*{claim}{Claim}                                         %%
 \newtheorem*{lemma*}{Lemma}                                        %%
 \newtheorem*{theorem*}{Theorem}                                    %%
 \newtheorem{lemma}[equation]{Lemma}                                %%
 \newtheorem{corollary}[equation]{Corollary}                        %%
 \newtheorem{proposition}[equation]{Proposition}                    %%
%%%%%%%%%%% End Theorem Styles and Counters %%%%%%%%%%%%%%%%%%%%%%%%%%

%%% These three lines load and resize a caligraphic font %%%%%%%%%
%%% which I use whenever I want lowercase \mathcal %%%%%%%%%%%%%%%
 \DeclareFontFamily{OT1}{pzc}{}                                 %%
 \DeclareFontShape{OT1}{pzc}{m}{it}{<-> s * [1.100] pzcmi7t}{}  %%
 \DeclareMathAlphabet{\mathpzc}{OT1}{pzc}{m}{it}                %%
                                                                %%
%%% and this is manfnt; used to produce the warning symbol %%%%%%%
 \DeclareFontFamily{U}{manual}{}                                %%
 \DeclareFontShape{U}{manual}{m}{n}{ <->  manfnt }{}            %%
 \newcommand{\manfntsymbol}[1]{%                                %%
    {\fontencoding{U}\fontfamily{manual}\selectfont\symbol{#1}}}%%
%%%%%%%%%%%%%%%%%%%%%%%%%%%%%%%%%%%%%%%%%%%%%%%%%%%%%%%%%%%%%%%%%%

%%%%%%%%%%%%%%%% Anton's Macros %%%%%%%%%%%%%%%%%%%%%%%%%%%%%%%%%%%%%%
 \def\<{\langle}
 \def\>{\rangle}
 \newcommand{\anton}[1]{[[\ensuremath{\bigstar\bigstar\bigstar} #1]]}
 \renewcommand\1{\mathbbm{1}}
 \renewcommand{\AA}{\mathbb{A}}
 \newcommand\A{\text{\sf A}}
 \newcommand{\ab}{\text{\sf Ab}}
 \newcommand{\Ad}{\mathrm{Ad}}
 \newcommand{\ad}{\mathrm{ad}}
 \newcommand{\alg}{\text{\sf Alg}}
 \newcommand{\aff}{\text{\sf Aff}}
 \DeclareMathOperator{\ann}{ann}
 \DeclareMathOperator{\aut}{Aut}
 \DeclareMathOperator{\Aut}{Aut}
 \newcommand{\B}{\text{\sf B}}
 \def\b{\text{\sf B}}
% \newcommand{\bbar}[1]{\overline{#1}}
 \newcommand{\bbar}[1]{\setbox0=\hbox{$#1$}\dimen0=.2\ht0 \kern\dimen0 \overline{\kern-\dimen0 #1}}
 \DeclareMathOperator{\ber}{Ber}
 \newcommand{\CC}{\mathbb{C}}
 \newcommand{\C}{\text{\sf C}}
 \newcommand{\cat}{\text{\sf Cat}}
 \DeclareMathOperator{\ch}{ch}
 \newcommand{\chain}{\text{\sf Chain}}
 \DeclareMathOperator{\codim}{codim}
 \newcommand{\coh}{\text{\sf Coh}}
 \DeclareMathOperator{\coker}{coker}
 \DeclareMathOperator{\colim}{colim}
 \newcommand{\csalg}{\text{\sf cSAlg}}
 \newcommand{\cw}{\text{\sf CW}}
 \newcommand{\D}{\text{\sf D}}
 \newcommand{\DDelta}{\Delta}
 \newcommand\der[2]{\frac{d #1}{d #2}}
 \DeclareMathOperator{\Der}{\sf Der}
 \let\div\relax % old \div is the "divides sign"
 \DeclareMathOperator{\div}{div}
 \newcommand{\e}{\varepsilon}
 \DeclareMathOperator{\End}{End}
 \newcommand{\E}{\mathcal{E}}
 \DeclareMathOperator\ext{Ext}
 \newcommand{\F}{\mathcal{F}}
 \newcommand{\FF}{\mathbb{F}}
 \newcommand\fr{\text{\sf Fr}}
 \newcommand\fun{\text{\sf Fun}}
 \newcommand\fsalg{\text{\sf FSAlg}}
 \newcommand{\g}{\mathfrak{g}}
 \newcommand{\G}{\mathcal{G}}
 \newcommand{\Ga}{\Gamma}
 \newcommand{\ga}{\gamma}
 \newcommand{\GG}{\mathbb{G}}
 \newcommand{\gl}{\mathfrak{gl}}
 \newcommand{\gpoid}{\text{\sf Gpoid}}
 \newcommand\gring{\text{\sf GRing}}
 \newcommand\gp{\text{\sf Gp}}
 \newcommand\gvect{\text{\sf GVect}}
 \newcommand{\h}{\mathfrak{h}}
 \renewcommand{\H}{\mathcal{H}} % old \H{x} is an x with a weird umlaut in text mode
 \newcommand{\HH}{\mathbb{H}}
 \DeclareMathOperator{\hilb}{Hilb}
 \newcommand\hchain{\text{\sf h-Chain}}
 \newcommand{\hhat}[1]{\widehat{#1}}
 \let\hom\relax % kills the old hom, which is lowercase
 \DeclareMathOperator{\hom}{Hom}
 \DeclareMathOperator{\Hom}{\mathcal{H}\kern-.125em\mathpzc{om}}
 \DeclareMathOperator{\HOM}{HOM}
 \renewcommand{\labelitemi}{--}  % changes the default bullet in itemize
 \newcommand{\I}{\mathcal{I}}
 \DeclareMathOperator{\id}{id}
 \DeclareMathOperator{\im}{im}
 \newcommand{\inn}[1]{\accentset{\circ}{#1}}
 \newcommand{\irr}{\mathrm{irreg}}
 \DeclareMathOperator{\isom}{\underline{Isom}}
 \newcommand{\J}{\mathcal{J}}
 \renewcommand{\k}{\mathfrak{k}}
 \newcommand{\K}{\mathcal{K}}
 \newcommand{\kan}{\text{\sf Kan}} 
 \renewcommand{\L}{\mathcal{L}}
 \DeclareMathOperator{\lie}{Lie}
 \newcommand{\lotimes}{\overset{L}{\otimes}}
 \newcommand{\m}{\mathfrak{m}}
 \newcommand{\M}{\mathcal{M}}
 \DeclareMathOperator{\mat}{Mat}
 \newcommand{\matx}[1]{\bigl(\begin{smallmatrix} #1 \end{smallmatrix}\bigr)}
 \newcommand{\Matx}[1]{\begin{pmatrix}#1\end{pmatrix}}
 \newcommand{\man}{\text{\sf Man}}
 \renewcommand{\mod}{\textrm{-\sf mod}}
 \DeclareMathOperator{\mor}{Mor}
 \newcommand{\nn}{\mathfrak{n}}
 \newcommand{\N}{\mathcal{N}}
 \DeclareMathOperator{\nat}{Nat}
 \DeclareMathOperator{\Nat}{NAT}
 \newcommand{\NN}{\mathbb{N}}
 \renewcommand{\O}{\mathcal{O}}
 \newcommand{\om}{\omega}
 \newcommand{\Om}{\Omega}
 \newcommand{\p}{\mathfrak{p}}
 \newcommand{\pb}{\rule{.4pt}{5.4pt}\rule[5pt]{5pt}{.4pt}\llap{$\cdot$\hspace{1pt}}}
 \newcommand{\po}{\rule{5pt}{.4pt}\rule{.4pt}{5.4pt}\llap{$\cdot$\hspace{1pt}}}
 \renewcommand{\P}{\mathcal{P}}
 \newcommand\pder[2]{\frac{\partial #1}{\partial #2}}
 \newcommand{\PP}{\mathbb{P}}
 \DeclareMathOperator{\pic}{Pic}
 \newcommand{\pre}{\mathrm{pre}}
 \DeclareMathOperator{\presh}{\sf PreSh}
 \DeclareMathOperator{\proj}{Proj}
 \DeclareMathOperator{\Proj}{\mathcal{P}\kern-.125em\mathpzc{roj}}
 \newcommand{\qbinom}[2]{\left[\frac{#1}{#2}\right]_q}
 \newcommand{\qco}{\text{\sf Qcoh}}
 \newcommand{\QQ}{\mathbb{Q}}
 \newcommand{\qft}{\mathit{QFT}}
 \newcommand{\quot}{\mbox{\rm /\!\!/}}
 \renewcommand{\r}{\mathfrak{r}} % old \r{x} is a little circle over x in text mode
 \newcommand{\rmod}{\text{\sf mod-}}
 \newcommand\rb{\text{\sf RB}}
 \newcommand{\reg}{\mathrm{reg}}
 \newcommand\riem{\text{\sf Riem}}
 \newcommand{\ring}{\text{\sf Ring}}
 \DeclareMathOperator{\rk}{rk}
 \newcommand{\RR}{\mathbb{R}}
 \let\SS=\S
 \renewcommand{\S}{\mathcal{S}}
 \newcommand{\salg}{\text{\sf SAlg}}
 \DeclareMathOperator{\sch}{\sf Sch}
 \DeclareMathOperator{\sdim}{sdim}
 \DeclareMathOperator{\set}{\sf Set}
 \renewcommand{\setminus}{\smallsetminus}
 \DeclareMathOperator{\sh}{\sf Sh}
 \newcommand\slie{\text{\sf SLie}}
 \renewcommand{\sl}{\mathfrak{sl}}
 \newcommand{\sman}{\text{\sf SMan}}
 \DeclareMathOperator{\spec}{Spec}
 \DeclareMathOperator{\Spec}{\mathcal{S}\!\mathpzc{pec}}
 \DeclareMathOperator{\specm}{Specm}
 \DeclareMathOperator{\spin}{Spin}
 \DeclareMathOperator{\str}{str}
 \DeclareMathOperator{\supp}{Supp}
 \newcommand{\su}{\mathfrak{su}}
 \newcommand\svect{\text{\sf SVect}}
 \DeclareMathOperator{\sym}{Sym}
% \renewcommand\t{\tau}  % old \t is something to do with accents
 \renewcommand{\t}{\mathfrak{t}}
 \newcommand{\T}{\mathcal{T}}
 \newcommand\tft{\mathit{TFT}}
 \newcommand{\tmf}{\mathit{TMF}}
 \renewcommand{\top}{\text{\sf Top}} % old \top is \mathchar"23E
 \DeclareMathOperator{\topoi}{\sf Topoi}
 \DeclareMathOperator{\tor}{Tor}
 \DeclareMathOperator{\tors}{\sf Tors}
 \DeclareMathOperator{\tot}{Tot}
 \DeclareMathOperator{\Tot}{\ensuremath{\mathpzc{Tot}}}
 \DeclareMathOperator{\tr}{tr}
 \newcommand{\TT}{\mathbb{T}}
 \newcommand{\ttilde}[1]{\widetilde{#1}}
 \newcommand{\udot}{\ensuremath{{\lower .183333em \hbox{\LARGE \kern -.05em$\cdot$}}}}
 \newcommand{\uudot}{{\ensuremath{\!\mbox{\large $\cdot$}\!}}}
 \DeclareMathOperator{\uhom}{\underline{Hom}}
 \newcommand{\U}{\mathcal{U}}
 \newcommand{\uC}{\underline{\C}}
 \newcommand{\unst}{\mathrm{unst}}
 \DeclareMathOperator{\umor}{\underline{Mor}}
 \newcommand{\V}{\mathcal{V}}
 \newcommand\vect{\text{\sf Vect}}
 \newcommand\VV{\mathbb{V}}
 \newcommand{\w}{\omega}
 \newcommand{\W}{\mathcal{W}}
 \newcommand{\X}{\mathcal{X}}
 \newcommand{\Y}{\mathcal{Y}}
 \newcommand{\Z}{\mathcal{Z}}
 \newcommand{\ZZ}{\mathbb{Z}}
%%%%%%%%%%%% End Anton's Macros %%%%%%%%%%%%%%%%%%%%%%%%%%%%%%%%%%%%%%%

 \openout0=lastupdated.html
 \write0{\today}
 \closeout0
 
