\sektion{2}{Matching}

\begin{definition}[Matching]
    A matching in a graph is a set of pairwise non-adjacent edges.
\end{definition}

\begin{definition}[Matching Number]
    The matching number of a graph G, denoted $\nu(G)$, is the size of a largest matching.
\end{definition}

\begin{proposition}
    $\nu(G) \leq \frac{n}{2}$
\end{proposition}

\begin{definition}[Perfect Matching]
    A perfect matching is a matching of size $\frac{n}{2}$.
\end{definition}

\[
    \nu(K_n) = \left\{\begin{array}{lr}
        \frac{n}{2},    &   \text{n is even}\\
        \frac{n-1}{2},  &   \text{n is odd}
    \end{array}\right\}
\]

\[
    (m \leq n) \nu(K_{m,n}) = m
\]

Questions
\begin{itemize}
    \item What can you say about nu(G) if G is Hamiltonian? n/2 or (n-1)/2
   
    \item What can you say about nu(G) if G is cubic and Hamiltonian? n/2
    \item What can you say about Chi(G) if G has no triangles? Nothing (Mycielski's theorem)
    \item Does every regular graph have a perfect matching?
    \item Will the properties: regular, even num of vertices, and connected force the existence of perfect matching?
\end{itemize}

\begin{definition}
    An independent set in a graph is a set of pairwise non-adjacent vertices.
\end{definition}

\begin{definition}
    An independence number of a graph, G, is the size of the largest independent set. This is denoted by $\alpha(G)$. 
\end{definition}
