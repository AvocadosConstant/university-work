\documentclass{article}

\usepackage{fancyhdr}
\usepackage{extramarks}
\usepackage{amsmath}
\usepackage{amsthm}
\usepackage{amsfonts}
\usepackage{tikz}
\usepackage{enumerate}
\usepackage{mathtools}
\usepackage{enumitem}
\usepackage{pgfplots}
\usepackage{diagbox}

%
% Basic Document Settings
%

\topmargin=-0.45in
\evensidemargin=0in
\oddsidemargin=0in
\textwidth=6.5in
\textheight=9.0in
\headsep=0.25in

\linespread{1.1}

\pagestyle{fancy}
\lhead{\hmwkAuthorName}
\chead{\hmwkClass\ (\hmwkClassInstructor\ \hmwkClassTime): \hmwkTitle}
\rhead{\firstxmark}
\lfoot{\lastxmark}
\cfoot{\thepage}

\renewcommand\headrulewidth{0.4pt}
\renewcommand\footrulewidth{0.4pt}

\setlength\parindent{0pt}

%
% Create Problem Sections
%

\newcommand{\enterProblemHeader}[1]{
    \nobreak\extramarks{}{Problem \arabic{#1} continued on next page\ldots}\nobreak{}
    \nobreak\extramarks{Problem \arabic{#1} (continued)}{Problem \arabic{#1} continued on next page\ldots}\nobreak{}
}

\newcommand{\exitProblemHeader}[1]{
    \nobreak\extramarks{Problem \arabic{#1} (continued)}{Problem \arabic{#1} continued on next page\ldots}\nobreak{}
    \stepcounter{#1}
    \nobreak\extramarks{Problem \arabic{#1}}{}\nobreak{}
}

\setcounter{secnumdepth}{0}
\newcounter{partCounter}
\newcounter{homeworkProblemCounter}
\setcounter{homeworkProblemCounter}{1}
\nobreak\extramarks{Problem \arabic{homeworkProblemCounter}}{}\nobreak{}

%
% Homework Problem Environment
%
% This environment takes an optional argument. When given, it will adjust the
% problem counter. This is useful for when the problems given for your
% assignment aren't sequential. See the last 3 problems of this template for an
% example.
%
\newenvironment{homeworkProblem}[1][-1]{
    \ifnum#1>0
    \setcounter{homeworkProblemCounter}{#1}
    \fi
    \section{Problem \arabic{homeworkProblemCounter}}
    \setcounter{partCounter}{1}
    \enterProblemHeader{homeworkProblemCounter}
    }{
    \exitProblemHeader{homeworkProblemCounter}
}

%
% Homework Details
%   - Title
%   - Due date
%   - Class
%   - Section/Time
%   - Instructor
%   - Author
%

% TODO: Replace with correct number and date
\newcommand{\hmwkTitle}{Problem Set\ \#1}
\newcommand{\hmwkDueDate}{September 18, 2017}
\newcommand{\hmwkClass}{CS 427}
\newcommand{\hmwkClassInstructor}{Professor Zerksis D. Umrigar}
\newcommand{\hmwkAuthorName}{Tim Hung}

%
% Title Page
%

\title{
    \vspace{2in}
    \textmd{\textbf{\hmwkClass:\ \hmwkTitle}}\\
    \normalsize\vspace{0.1in}\small{Due\ on\ \hmwkDueDate\ at 5:50pm}\\
    \vspace{0.1in}\large{\textit{\hmwkClassInstructor\ \hmwkClassTime}}\\
}

\author{\textbf{\hmwkAuthorName}}
\date{}

\renewcommand{\part}[1]{\textbf{\large Part \Alph{partCounter}}\stepcounter{partCounter}\\}

%
% Various Helper Commands
%

% Useful for algorithms
\newcommand{\alg}[1]{\textsc{\bfseries \footnotesize #1}}

% For derivatives
\newcommand{\deriv}[1]{\frac{\mathrm{d}}{\mathrm{d}x} (#1)}

% For partial derivatives
\newcommand{\pderiv}[2]{\frac{\partial}{\partial #1} (#2)}

% Integral dx
\newcommand{\dx}{\mathrm{d}x}

% Alias for the Solution section header
\newcommand{\solution}{\textbf{\large Solution}}

% Probability commands: Expectation, Variance, Covariance, Bias
\newcommand{\E}{\mathrm{E}}
\newcommand{\Var}{\mathrm{Var}}
\newcommand{\Cov}{\mathrm{Cov}}
\newcommand{\Bias}{\mathrm{Bias}}

\begin{document}

\maketitle

\pagebreak

\begin{homeworkProblem}[]

    Discuss how you would extend the implementation of tl0 discussed
    in class to support a unary minus operator to allow evalution of
    expressions like `1 - -2`.  It is not necessary to show code but
    sufficient to precisely describe how you would change the
    implementation. ``15-points".\\

    \textbf{Solution} 

\end{homeworkProblem}

\begin{homeworkProblem}[]

    I once used a proprietary programming language where the programming
    language manual had a strange restriction.  It went something like:\\

    \\

    ``Integers can have a value between -2147483648 through 2147483647
    inclusive.  However, the literal value -2147483648 cannot be
    present in the program; all other integer literal values can
    be present in the program."\\

    Discuss possible reasons for this strange restriction.\\

    \\

    *Hint*: Our implementation of tl0 is subject to similar
    restrictions but not our implementation of tl1 (the difference is
    *not* due to our use of Ruby for implementing tl1). ``15-points"\\

    \textbf{Solution} 

\end{homeworkProblem}

\begin{homeworkProblem}[]

    Write regular expressions for each of the following: 

    Your answers should use the regex syntax discussed in class. ``15-points"

    \begin{enumerate}[label=(\alph*)]
        \item Strings over $\{a, b, c\}$ with an even number of $a$'s.
        \item
            Strings delimited using double quote characters ", where a
            string can contain any character except double quotes and can
            even contain double quotes if such contained double quotes are
            doubled.  Examples of legal strings include the string "" of
            length 0, the string ``a" of length 1, and the string """"
            of length 1 containing a single double-quote character.
        \item
            Positive binary numbers $n$ without leading zeros such that
            there exist positive integers $a$, $b$ and $c$ with $a^n +
            b^n = c^n$.  
    \end{enumerate}
    
\end{homeworkProblem}

\begin{homeworkProblem}[]

    In a string of length $n$, how many of the following are there?

    \begin{enumerate}[label=(\alph*)]
        \item prefixes

            Every substring that starts at index 0 and ends at index $i\in \{1, n\}$ is a prefix.
            \[n\]
        \item substrings

            For each substring, we choose a start index and an end index.

            There are $n\choose{2}$ ways to choose two indices from n characters.

            \[n\choose{2}\]
        \item subsequences

            A subsequence is comprised of any element of the powerset of the string.

            The size of the powerset of a set of size n is
            \[2^n\]
    \end{enumerate}
    
\end{homeworkProblem}

\begin{homeworkProblem}[]

    Why would it be a problem if a programming language had 2
    different binary operators $\oplus$ and $\otimes$ having
    the same precedence, but $\oplus$ was left associative
    while $\otimes$ was right associative? ``0-points"\\

    \textbf{Solution} 

\end{homeworkProblem}

\begin{homeworkProblem}[]

    Write ``regular definitions" to describe the following languages:

    Your answers should use the regex syntax discussed in class.  ``15-points"\\

    \textbf{Solution} 

    \begin{enumerate}[label=(\alph*)]
        \item 
            All strings of balanced parentheses with a maximum nesting
            depth of 2.  Examples include $\epsilon$, (), (())() and
            (()())().
        \item 
            All strings of balanced parentheses.  Example include
            $\epsilon$, (), (())() and (((()())))().

        \item 
            C floating point constants as defined on pg 194 of the
            ``The C Programming Language" book:

            A floating constant consists of an integer part, a decimal
            point, a fraction part, an `e` or `E`, an optionally signed
            integer exponent and an optional type suffix, one of `f`,
            `F`, `l` or `L`.  The integer and fraction parts both
            consist of a sequence of digits.  Either the integer part
            or the fraction part (not both) may be missing; either the
            decimal point or the `e` and the exponent (not both) may be
            missing. 
    \end{enumerate}
    
\end{homeworkProblem}

\begin{homeworkProblem}[]

    Consider the Java java.util.regex package:\\

    \textbf{Solution} 

    \begin{enumerate}[label=(\alph*)]
        \item 
            It provides 3 different kinds of quantifiers: ``greedy
            quantifiers", ``reluctant quantifiers" and ``possesive
            quantifiers".  Give situations where each of these different
            types of quantifiers may be useful.
        \item 
            # How would you use this package to write a scanner for a
            programming language.  Consider as an example, a language
            which contains tokens for numbers, identifiers and strings,
            with whitespace.and comments ignored (the exact details of
            the lexical syntax being unimportant).  Your proposal should
            specifically address the following:

            \begin{itemize}
            \item 
                How would you set things up so that for each recognized
                lexeme you would only make a single call to the Java RE
                engine?

            \item
                After the call to the Java RE engine completes, how would
                you identify which kind of token has been matched?

            \item 
                Which of the several calls in the `Matcher` class would you
                use to do the matching?

            \item
                How would you structure a `nextToken()` function?
            \end{itemize}
    \end{enumerate}
    ``15-points"
\end{homeworkProblem}

\end{document}
